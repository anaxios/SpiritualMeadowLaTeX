146, THE VISION OF EULOGIOS,
PATRIARCH OF ALEXANDRIA

'When we were at the Community of Tougara, nine miles outside
Alexandria, Abba Menas who ruled that community,* told us this
concerning the saintly Pope Eulogios:

One night when he was performing the office alone in the chapel of
the episcopal residence, he saw the Archdeacon Julian standing
before him.
When he saw him he was disturbed that the man should
have dared to enter unannounced, but he said nothing.
At the end
of the psalm, he prostrated himself; and so too did the one who had
appeared to him in the form of the archdeacon.
When the pope got
up and offered the prayer, the other one remained prostrate on the
ground, The pope turned to him and said: 'How long will it be
before you stand up?' The other said: 'Unless you offer me your
hand and raise me, I cannot stand up'.
Then the abba put out his
hand, took hold of him and raised him up.
Then he took up the
psalm again; but when he turned round, he no longer saw anybody.
When he had completed the dawn office, he called for his chamber-
lain and said to him: 'Why did you not announce the entry of the
archdeacon, but Jet him come to me unannounced, and that in the
night-time?' The chamberlain said neither had he seen anybody nor
had anybody come in.
The pope was not convinced.
'Call the porter
here,' he cried, and when the porter arrived he said to him: 'Did the
Archdeacon Julian not come in here?' The porter asserted with an
oath that the archdeacon had neither come in nor gone out.
Then
the pope kept his peace.
When day dawned, Archdeacon Julian
came in to pray.
The pope said to him: 'Why did you break the rule
by coming in to me unannounced last night, Archdeacon Julian?' He
replied: 'By the prayers of my lord, I did not come in here last

night, nor did I leave my own house until this very hour'.
Then the
great Eulogios realised that it was Julian the Martyr he had seen,
urging him to <re>build his church which had been dilapidated for
some time and antiquated, threatening to fall down.
The godly
Eulogios, the friend of martyrs, set his hand to the task with
determination, By rebuilding the martyr's temple from its founda-
tions and distinguishing it with a variety of decoration, he provided
a shrine worthy of a holy martyr.

