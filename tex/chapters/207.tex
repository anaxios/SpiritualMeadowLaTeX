207. THE LIFE OF AN ALEXANDRINE GIRL
WHO WAS RECEIVED FROM THE SACRED FONT
BY ANGELS

Abba Theonas and Abba Theodore said that in the time of the
Patriarch Paul, there was a maiden in Alexandria who lost both her
parents—and they possessed a great fortune. The girl was
unbaptised at the time <of her bereavement>. One day she went
apart into the garden which her parents had left her (for there are
gardens in the middle of the city, in the houses of the great ones).
Whilst she was in the garden, she saw a man preparing to hang
himself. She rushed to him and said: 'What are you doing, good
man?' He said to her: 'Look, leave me alone woman, for I am in

great affliction'. The maiden said to him; 'Tell me the truth, for
perhaps I may be able to help you'. He told her: 'I am heavily in
debt and my creditors are putting pressure on me to repay them. I
have chosen to die rather than to lead such a woeful existence.' The
maiden said to him: 'I beg of you, take whatever I have and give it
to them; only please do not destroy yourself. He took what she
offered and paid off his debts. Then the girl began to run into
difficulties, Having no one to look after her (because she had been
deprived of her parents) and being in great need, she began to
prostitute herself. Some people who knew her, and knew the
standing which her parents had enjoyed in society, said: 'Who
knows the judgements of God or why he allows a soul to fall for
some reason or other?' The some time later, the girl fell ill—and
came back to her senses. Consumed with remorse, she said to her
neighbours: 'For the sake of the Lord, have mercy on my soul;
speak to the pope about making me a Christian'. But they all
laughed at her and said: 'As if he would accept this woman who is
a prostitute!' This caused her great distress. Whilst she was in this
condition and very frustrated, an angel of the Lord stood by her—in
the form of the man on whom she had compassion. He said to her:
'What is the trouble?" She replied: 'I desire to become a Christian
and nobody will stand up for me'. He said: 'Do you really want
this?' She replied: 'Yes, I beg of you'. He said to her: 'Take courage;
I will get some people to take you to church'. He brought two
others who were also angels and they carried her to the church.
Then they transformed themselves into illustrious personages with
the rank of prefect. They summoned the clergy charged with the
responsibility for baptisms, and these asked: 'Your Charity will
vouch for her? They answered: 'Yes'. Then the clergy did what was
called for in the service for those who are about to be baptised; then
they baptised her in the name of the Father and of the Son and of
the Holy Spirit, and they vested her in the garment of the neophyte.
Clothed in white, she returned home carried by the angels, who set

her down and promptly disappeared. When the neighbours saw her
all in white, they said to her: 'Who baptised you?'—and she told
them of those who had taken her to church, how they had spoken
to the clergy and how the clergy had baptised her. They asked her
who those people were; to which the woman would give no answer.
So they went and reported the matter to the pope. He summoned
those in charge of the baptistry and said to them: 'Did you baptise
that woman?' They admitted that they had <baptised her>, adding
that she had been vouched for by so-and-so of prefectorial rank,
The bishop sent for those whom they had named and enquired of
them whether they had vouched for her. They said: “We are not
aware of having done so, nor do we know anybody else who has'.
Then the bishop realised that this was divine business. He sum-
moned the woman and said: 'Tell me, daughter, what good have
you done?' She said: 'I am a prostitute and a poor woman too; what
good could I do?' He said to her: 'Are you not aware of ever having
done any good <deed> at all?' She said: 'No. Except that I once
saw a man about to hang himself because he was being harassed by
his creditors. I gave him my entire fortune and freed him <of his
debt>.' She said this, and fell asleep in the Lord, released from both
her voluntary and her involuntary deeds of sin. Then the bishop
glorified God and said: Righteous you are O God, and upright are
your judgements <Ps 118:137.>

