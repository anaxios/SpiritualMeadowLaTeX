42, THE LIFE OF ABBA AUXANON

At the same place we saw Abba Auxanén in his cell, a man of
compassion, continence and solitude who treated himself so harshly
that over a period of four days he would only eat a twenty-/epta*
loaf of bread, such as we offer at the eucharist. Sometimes this was
sufficient for him during a whole week. Towards the end of his life,

this ever-memorable father fell ill with a stomach complaint. They
carried him to the patriarchal infirmary at the Holy City. One day
when we were visiting him, Abba Conon, higoumen of the Lavra of
our saintly father Sabas, sent him a small basket containing the
church dole and six pieces of gold, with a message: 'Forgive me, but
my sickness prevents me from coming to greet you'. The elder
accepted the dole but sent the gold back to him with this message:
'If it be the will of God that I be in this life, father, I have ten
pieces of gold. If I have need of these others I will let you know,
and do you send them to me. But you should know, father, that
two days from now I will go forth out of this world'—which indeed
he did. We bore him to the Lavra of Pharén and buried him there.
This blessed one had been the fellow-monk of those saintly men
Eustochios and Gregory but, leaving them both, he completed his
formation in the wilderness.* He was a native of Ancyra in Galatia.

