69.
THE LIFE OF ABBA PALLADIOS AND OF AN
ELDER OF THESSALONICA,
A RECLUSE NAMED DAVID

Master Sophronios the sophist (before he embraced the monastic
life) and I met Abba Palladios in Alexandria.
He was a man who
both loved and served God and he had his monastery at Lithazo-
menon.
We pressed him to speak an edifying word to us.
The elder
began to say to us: 'Children, the time that remains to us is short.
Let us struggle for a little <in this world> and Jabour, in order that
we might have the enjoyment of very great things in eternity.
Look
at the martyrs, look at the holy men, look at the ascetics; see how
courageously they persevered.
We will ever wonder at the endurance
of those whose remembrances have been preserved from time past.
Every one who hears of them acknowledges with great astonishment
the spperhusnan endurance of the blessed martyrs; how their eyes
were plucked out; how the legs of some of them were cut off, others
their hands, whilst some had their feet destroyed.
How some were
eliminated by raging fire whilst others were slowly roasted.
How
some were drowned in rivers, others at sea.
How some were torn
apart by carnivorous beasts like criminals whilst others were fed to
birds of prey after suffering exquisite tortures.
In brief, if it were
possible to describe all the different tortures which were devised for
their affliction, everything that the enemy, the devil, has inflicted
upon the martyrs and ascetics who loved God, it would be seen how
much they endured and how they have wrestled, triumphing over
the weakness of the flesh by the courage of the soul.
They attained
to those good things for which they hoped by counting them more
worthy than the trials of this earthly life.
This provides a demon-
stration of the solid quality of their faith in two ways.
On the one

hand, that having endured a little, they now enjoy great benefits in
eternity.
On the other hand, that they so cheerfully endured the
physical torments with which the adversary the devil afflicted them.
If therefore we endure affliction and persevere, with the help of
God, we shall be found to be friends of God indeed.
And God will
be with us, fighting shoulder to shoulder with us in the battle,
greatly alleviating that which we must endure.
My children, since we
know what kind of times these are and what kind of labour is
required of us, let us strive for the self-knowledge which is attained
by means of the solitary life.
For at this stage it is required of us
that we sincerely repent, so that we may indeed be temples of God.
For it will not be honour such as the world gives that we will
receive in the world to come'.

Again he said: 'Let us remember Him who has nowhere to lay
his head', <Mt 8:20> and again: 'Since Saint Paul says Trbu/ation
worketh patience, <Rm 5:3> let us make our minds able to receive
the kingdom of heaven'.
And again: 'Children, Jet us not Jove the
world, neither those things which are in the world' <\ Jn 2:15.>
Again the elder said: “Let us keep a guard over our thoughts, for
this is the medicine of salvation'.

We went to the same Abba Palladios with this request; 'Of your
charity, tell us, father, where you came from, and how it came
about that you embraced the monastic life'.
He was from Thessa-
lonica, he said, and then he told us this: 'In my home country,
about three stades beyond the city wall, there was a recluse, a native
of Mesopotamia whose name was David.
He was a man of
outstanding virtue, merciful and continent.
He spent about twenty
years in his place of confinement.
Now at this time, because of the
barbarians, the walls of the city were patrolled at night by soldiers.
One night those who were on guard-duty at that stretch of the city-
walls nearest to where the elder's place of confinement was located,
saw fire pouring from the windows of the recluse's cell.
The soldiers
thought the barbarians must have set the elder's cell on fire; but

when they went out in the morning, to their amazement, they found
the elder unharmed and his cell unburned.
Again the following night
they saw fire, the same way as before, in the elder's cell—and this
went on for a long time.
The occurrence became known to all the
city and <throughout> the countryside.
Many people would come
and keep vigil at the wall all night long in order to see the fire,
which continued to appear until the elder died.
As this phenomenon
did not merely appear once or twice but was often seen, I said to
myself: 'If God so glorifies his servants in this world, how much
more so in the world to come when He shines upon their face like
the sun? This, my children, is why I embraced the monastic life.'

