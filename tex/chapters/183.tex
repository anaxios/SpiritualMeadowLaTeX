183, THE WONDROUS DEED OF DAVID,
THE EGYPTIAN

Abba Theodore the Cilician said:

When I was staying at Scété, there was an elder there called David.
One day he went out with some other monks to reap.
The Scetiotes
have this custom, that they go out to the estates and reap.
The elder
went to an estate and offered himself for hire an a day-to-day basis.
A farmer hired him and as the elder was reaping <..> about the
sixth hour it was very hot, so the elder entered a shack and sat
down, When the farmer came and saw him sitting there, he said to
him angrily: 'Elder, why are you not reaping? Do you not realise
that I am paying you?' He said: 'Yes, but the heat is so intense that
the grains of wheat are falling out of the husks.
I am waiting a lite
for the heat to abate so that you suffer no loss.' The farmer said to
him: 'Get up and work, even if everything bursts into flames'.
The
elder said to him: 'Do you want it all to burn?' The farmer angrily
rejoined: 'Isn't that what I said?' The elder stood up, and suddenly

the field began to burn.
Then, in fear, the farmer came to the other

part of the field where the other elders were reaping.
He begged
them to come and intercede with the elder for him, to pray that the
fire might cease.
They came and made an act of obeisance to the
elder, who said: 'But he himself said that it should burn'.
Yet they
were able to convince him, He went and stood between that part of
the field which was burning and that which remained unscathed.
He
offered a prayer and immediately the fire in the field was extin-
guished.
The rest of the crop was saved.
Everybody was amazed and
glorified God.

