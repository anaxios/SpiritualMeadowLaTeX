46. THE WONDROUS VISION OF ABBA CYRIACOS
OF THE LAVRA OF CALAMON AND
CONCERNING TWO BOOKS
OF THE IMPIOUS NESTORIOS

We once paid a visit to Abba Cyriacos the priest at the Lavra of
Calamén on the Holy Jordan and he told us this story:

One day, in my sleep, I saw a woman of stately appearance clad in
purple and after her <I saw> two reverend and honourable men
standing outside my cell. It seemed to me that the woman was our
Lady the Mother of God and that the men with her were Saint John
the Divine and Saint John the Baptist. I went out of my cell and
invited them to come in and offer a prayer in my cel], but she would
not agree <to my request.> I persisted at some length, entreating
her and saying: Oh Jet the simple not go away ashamed <Ps 73:21>
and much else. When she realised that I was importunate with my
invitation, she answered me coldly, saying: 'How can you ask me to
enter your cell when you have my enemy in there?' With these
words she went away. When I awoke, I began to worry and to
wonder if I might have offended her in my thoughts, for there was
nobody in the cell but me. I examined myself at some length and
could find no fault which I might have committed against her. As
it seemed that I was about to be overcome with remorse, I rose up
and took up a scroll, intending to read it, thinking that perhaps
reading would alleviate my distress. It was a book I had borrowed
from Hesychios, priest of Jerusalem. I unwound it and found two
writings of the irreligious Nestorios written at the end of it—and
immediately I knew that he was the enemy of our Lady, the holy
Mother of God, So I rose up and went off and gave the book back
to him who had given it to me. I said to him: 'Take your book
back, brother, for I have not derived as much benefit from it as it
has brought adversity upon me'. When he asked me how it had
caused me adversity, I told him what had happened. When he had

heard about it all, he immediately cut the writings of Nestorios off
from the scroll and threw the piece into the fire, saying: 'The enemy
of our Lady, the holy Mother of God, shall not remain in my cell

either'.

