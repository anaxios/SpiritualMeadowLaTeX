174. THE DEED OF A RELIGIOUS SHIP-MASTER
WHO PRAYED TO THE LORD FOR RAIN

Abba Gregory the anchorite told us:

I was returning from Byzantium by ship and a scribe came aboard
with his wife; he had to go pray at the Holy City. The ship-master
was a very devout man, given to fasting. As we sailed along, the
scribe's attendants were prodigal in their use of water. When we
came into the midst of the high sea, we ran out of water and we
were in great distress. It was a pitiful sight: women and children and
infants perishing from thirst, lying there like corpses. We were in
this distressing condition for three days and abandoned hope of
survival. Unable to tolerate such affliction, the scribe drew his
sword, intending to kill the ship-master and the sailors. He said: 'It
is their fault that we are to be lost, for they did not take sufficient
water on board for our needs'. I interceded with the scribe, saying:
'Do not do that; but rather, let us pray to our Lord Jesus Christ,
our true God, who does great and wonderful things which cannot
be counted <Jb 34:26>. Behold, this is now the third day that the
ship-master has occupied himself with fasting and prayer.' The
scribe quietened down and, on the fourth day, about the sixth hour,
the ship-master got up and cried in a loud voice: 'Glory to thee,
Christ our God'!—and that in such a way that we were all aston-
ished at his cry. And he said to the sailors: 'Stretch out the skins',
and whilst they were unfolding them, look! A cloud came over the
ship and it rained enough water to satisfy all our needs, It was a
great and fearful wonder, for as the ship was borne along by the
wind, the cloud followed us; but it did not rain beyond the ship.

