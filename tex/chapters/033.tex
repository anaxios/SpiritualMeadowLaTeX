33, THE LIFE OF THE HOLY BISHOP THEODOTOS

One of the fathers told us that there was formerly an archbishop of
Theoupolis* <=Antioch> whose goodness was such that once when
there was a feast-day he invited several of the clergy who had
celebrated the feast with him to dinner.
There was one of them who
refused the invitation.
The patriarch made no comment but on
another occasion he went personally to find the cleric and invited
him to share his table.
They also told this story about the same
Archbishop Theodotos: such were his humility and lowliness that
'once he was travelling with one of his clergy, the bishop <reclining>
in a litter whilst the cleric rode a horse.
The patriarch said to him:
'Let us defray the tedium of the journey by exchanging our modes
of travel', 'It would disgrace the patriarch' said the cleric, 'if I were
to get into the litter and he to mount the horse'.
The godly
Theodotos would have nothing of that.
He persuaded his attendant
cleric that it would be no disgrace and so prevailed upon him to
make the exchange.

