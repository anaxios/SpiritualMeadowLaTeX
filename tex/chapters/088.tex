88.
THE LIFE OF ABBA THOMAS, THE STEWARD OF
A COMMUNITY NEAR APAMEA AND THE MIRACLE
OF HIS CORPSE AFTER HE DIED

When we were at Theoupolis <=Antioch> a priest of the church
told us about the steward of a community in the district of Apamea,
Abba Thomas.
He came into Theoupolis to attend to the needs of
the monastery.
Whilst he was lingering there he died at Daphne, in
the Church of Saint Euphemia.
As he was a stranger, the local
clergy buried him in the strangers' burial-ground.
The following day
they buried a woman and laid her on top of him.
This was about
the second hour.
Around the ninth hour the earth threw her up.
'When the local people saw this they were amazed.
They buried her
again that evening in the same grave and next day they found her
remains on the top of the tomb, So they took the body and buried
it in another grave.
A few days later they buried another woman
and laid her above the monk, not realising that he would not allow
a woman to be buried on top of him.
When the earth threw up this
woman too, then they realised the fact that the elder would not
tolerate a woman being buried above him, Then they went to
Domninos the patriarch <546-559>.
He caused all the city to come
to Daphne with candles and with the singing of psalms, to bring
forth the relics of that holy man.
They buried him in the cemetery
where many relics of holy martyrs lie, and they built a small oratory
over him.

