180, THE LIFE OF JOHN THE ANCHORITE*
WHO LIVED IN A CAVE ON THE SOCHO ESTATE

The most holy Dionysios, priest and sacristan of the most holy
church of Ascalon, said to us concerning Abba John the Anchorite:
'This was a man who, in our own generation, was truly great in the
eyes of God', and as a demonstration of the extent to which he was
pleasing to God, he related this miracle attributed to him:

This elder lived in a cave in the district of the Socho estate, almost
twenty miles from Jerusalem. In the cave he had an icon of our all-
holy and spotless Lady, the Mother of God and ever-virgin Mary,
holding our God in her arms. Sometimes this elder would decide to
go somewhere. on a journey; maybe a great distance into the

wilderness, or to Jerusalem to reverence the Holy Cross and the
Holy Places, or to pray at Mount Sinai, or to visit martyrs
<shrines> many a long day's travel from Jerusalem. He was greatly
devoted to the martyrs, this elder. Now he would visit Saint John
at Ephesos; another time, Saint Theodore at Euchaita or Saint
Thecla the Isaurian at Seleucia or Saint Sergios at Saphas. Some-
times he would go to visit this saint, sometimes another. Yet
whenever he was about to set out, it was his custom to prepare and
light a lamp. He would stand in prayer, beseeching God to make
straight the way which-lay before him; running towards her icon, he
would say to the Lady: 'Holy Lady, Mother of God: since I am
about to undertake a Jong journey of many days' duration, watch
over your lamp and keep it from going out, as I intend that it
should not. For I am setting out with your help as my travelling-
companion.' Having said this to the icon, he would set off <on his
journey>. When he returned from his proposed trip, maybe a month
or two months or three months later, even sometimes after five or
six months, he would find the lamp well cared for and alight, just
as he had left it when he set out on his journey. He never saw it go
out of its own accord; not when he awakened from sleep or when
he returned to his cave from a journey or from the wilderness.

