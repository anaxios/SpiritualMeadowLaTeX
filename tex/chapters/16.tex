16. ABBA NICOLAS' STORY
ABOUT HIMSELF AND HIS FRIENDS

There was an elder living at the Lavra of Abba Peter* near the holy
Jordan whose name was Nicolas. He told us that when he was
staying at Raithou,* three of the brethren, <of whom he was one>
were sent to perform a service* at the Thebaid. “But when we were
going through the desert', he said, 'we lost our way and wandered.
far and wide. Our water was all used up and we went for days
without finding any. We began to faint from thirst and heat. When
we could not take one more step, we found some tamarisk trees
there in the desert and flung ourselves down wherever any shade
could be found, fully expecting to die of thirst. As I lay there I fell
into an ecstasy and I saw a pool of water full to overflowing. Two
people were standing at the edge of the pool, drawing water with a
wooden vessel. I began to make a request of one of them in these
words: 'Of your charity, sir, give me a little water, for I am faint',
but he was unwilling to grant my request. The other one said to
him: 'Give him a little', but he replied: 'No, let us not give him any,
for he is too easy-going,* and does not take care <of his soul'.> The
other said: 'Yes, yes; it is true that he is easy-going but he is
hospitable* to strangers',—and so he gave some to me and also to
my companions. We drank and went on our way, travelling three
more days without drinking until we reached civilisation.*

