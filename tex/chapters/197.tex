197.
RUFINUS' ANECDOTE OF SAINT ATHANASIOS
AND OTHER BOYS WHO WERE WITH HIM*

Rufinus, the ecclesiastical historian, reported something similar
which happened a long time ago to children at play.
It concerns
Saint Athanasios, the great proclaimer and defender of the truth, the
bishop of the great city of Alexandria, who shepherded all his
charges prudently and according to the will of God.
Speaking of the
saint's childhood, <Rufinus> shows how his elevation to the
episcopate was originally foreshadowed by a revelation to him from
God.
Let us trace the history of this man, the kind of life he led as

a child and the manner of his up-bringing, insofar as these things
have come to our ears.

The saintly Alexander succeeded Achilles as Pope of Alexandria,
just as Saint Peter the martyr-archbishop foretold, he who con-
demned the impious Arius.
One day, Alexander was looking out to
sea; he saw some children playing on the shore as children usually
do.
They were imitating a bishop and all the ceremonies which are
customary in church.
Paying careful attention to what was going on,
he realised that they were acting out some of the secret parts of the
mysteries.
This troubled him, so he immediately summoned the
clergy.
He showed them what was taking place and required them
to go and apprehend all those children and bring them to him.
When they arrived, he asked them about the nature of their game
and what they were doing.
Being children, they were frightened,
<and at first> they denied everything.
But then they told him every
detail of their game: how they had baptised some catechumens by
<the hand of> Athanasios—whom those children had appointed as
their bishop.
Then Alexander enquired diligently of them which ones
they had baptised and when he discovered that everything had been
performed strictly in accordance with the customs of our religion,
he informed his clergy of this, decreeing that those who had been
made worthy of that holy bath stood in no need of a second
baptism.
He sent back to their parents Athanasios and the others
who served as clergy, to be brought up in the fear and nurture of
the Lord; especially Athanasios, whom he very soon afterwards
consecrated to God.
Being better endowed with godly attributes, he
was advanced to a higher rank by the then-archbishop; such was his
distinction.
<The archbishop> summoned the parents* of Athansios
and of the other <children> whom the latter pretended to have as
his priests and deacons in the game and, with God as his witness,
handed them over to the church to be nourished therein.

A little time elapsed during which Athanasios was thoroughly
educated by a short-hand writer and well-enough by a grammar-

school teacher.
Then, as a sacred trust committed to them by the
Lord, he was handed back to the priest by his parents and, like a
second Samuel, he was raised in the temple of God.
And when
Alexander went to visit <other> bishops in his old age, he would
have Athansios follow him, carrying the vestment of priesthood
which is called ephod in the Hebrew tongue.

So great were Athansios' exertions against the heretics on behalf
of the Church that it might seem as though that verse were
especially written for him—the one which says: I will show him
what he must suffer for my name's sake.
The whole world conspired
to persecute him; the kings of the earth* showed him what he must
suffer for my name's sake.
The whole earth moved, kingdoms and
an army came together against him.
But he stood fast by the saying
of God which says: Though a host of men were laid against me, yet
shall not my heart be afraid.
And though there ros¢ up war against
\_me yet will I put my trust in him <Ps 26:3.> But so many and such
things are reported of him which are so important that they cannot
be passed over in silence.
Yet they are nevertheless so numerous that
I am compelled do so in may cases.
I am on the horns of a
dilemma, not being able to decide what to retain and what to let go.
That is why we are recording a few matters which are directly
connected with the subject; the rest of them will be relayed by
common report.
Common report, however, can be relied on to relay
less than the <whole> truth, for it has neither the ability nor
anything to add to the truth.

