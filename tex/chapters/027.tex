27.
THE LIFE OF THE PRIEST OF THE
MARDARDOS ESTATE

Ten miles from the city of A2gaion in Cilicia there is an estate called
Mardardos and on it there is an oratory of Saint John the Baptist.
There resided an elder who was a priest, an elder of great prestige
and virtue.
One day those who lived on that estate went to complain
about him to the Bishop of Aégaion.
'Take this elder away from us',
they said, 'for he is objectionable to us.
When Sunday comes
around, he holds the service at the ninth hour and even then he
does not follow the appointed order of service'.
The bishop took the
elder aside privately and said to him: 'Good elder, why do you
behave like this? Do you not know the procedure of holy church?'
The elder said to him: 'Truth to tell, great sir, it is just as you say
and you have spoken well.
But I do not know what to do.
After the
vigil-service of the holy Lord's Day I remain close by the holy altar;
and until I see the Holy Spirit over-shadowing the holy sanctuary,
I do not begin the <eucharistic> service.
When I see the coming
[éxdoftn orc] of the Holy Spirit, then I celebrate the liturgy'.
The bishop was amazed at the virtue of the elder.
He informed the
inhabitants of the estate and dismissed them in peace; they went
their way glorifying God.

Abba Julian the stylite sent greetings to this elder, sending him
a folded cloth with three coals of fire within.
The elder received the
greeting and the three coals stil] not dead.
He sent them back to
Abba Julian in the same cloth having poured water into it and tied
it up.
They were about twenty miles distant from each other.

