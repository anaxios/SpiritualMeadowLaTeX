100.
THE LIFE OF PETER, THE MONK OF PONTUS

Again the fathers of the same place told us that there had been a
priest there whose name was Peter, a native of Pontus, who did
many great and wondrous deeds.
Theodore (who became Bishop of
Rossos) told us something about this elder:

One day he came up to me at the Jordan, in the Pyrgia Lavra where
I was staying, and said to me: 'Brother Theodore, of your charity,
come up into Mount Sinai with me, for I have a prayer <to offer>.'
Not wishing to deny him, I said to him: 'Let us go'.
When we had
crossed the holy Jordan, the elder said to me: 'Brother Theodore, let
us offer this as an act of penitence: that neither of us will eat
anything until we come to Mount Sinai'.
I said to him: 'Truly,
father, that is more than I am capable of, so the elder made his

<own> resolution—and ate nothing until we came to Sinai.
At Sinai
he partook of the holy mysteries and then ate some food.
In the
same way, from Sinai to Saint Menas* at Alexandria, he ate no
food.
There too he received holy communion and then ate.
From
Saint Menas we went to the Holy City and he tasted nothing
whatsoever along the way.
He made his communion at the <Church
of> the Holy Resurrection of Christ our God and then took some
food.
In all that long journeying the elder only ate three times: once
at Mount Sinai, once at Saint Menas and once in the Holy City.

