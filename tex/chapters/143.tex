143.
DAVID, THE ROBBER-CHIEF, WHO LATER
BECAME A MONK

We came to the Thebaid and at the city of Antinoé we visited
Phoebamon the Sophist for the benefit <of his words>.
He told us
that in the district around Hermopolis there had been a brigand
whose name was David.
He had rendered many people destitute,
murdered many and committed every kind of evi] deed; more so
than any other man, one might say.
One day, whilst he was still
engaged in brigandage on the mountain, together with a band of
more than thirty, he came to his senses, conscience-stricken by his
evil deeds.
He left all those who were with him and went to a
monastery.
He knocked at the monastery gate; the porter came out
and asked him what he wanted.
The robber-chief replied that he
wanted to become a monk, so the porter went inside and told the
abbot about him.
The abbot came out and, when he saw that the
man was advanced in age, he said to him: 'You cannot stay here,
for the brethren labour very hard.
They practice great austerity.
'Your temperament is different from ours and you could not tolerate
the rule of the monastery.' But the brigand insisted that he could
tolerate these things, if only the abbot would accept him.
But the
abbot was persistent in his conviction that the man would not be
able.
Then the robber-chief said to him: 'Know, then, that I am

David, the robber-chief, and the reason why I came here was that
I might weep for my sins.
If you do not accept me, I swear to you
and before him who dwells in heaven that I will return to my
former way of life.
I will bring those who were with me, kill you all
and even destroy your monastery.' When the abbot heard this, he
received him into the monastery, tonsured-him and gave him the
holy habit.
Thus he began the spiritual combat and he exceeded all
the other members of the monastery in self-control, obedience and
humility.
There were about seventy persons in that monastery; he
benefitted them all, providing them with an example.

One day when he was sitting in his cell, an angel of the Lord
appeared to him, saying: 'David, David; the Lord has pardoned
your sins and, from this time on, you shall perform wonders
<sémeia>.' David replied to the angel: 'I cannot believe that in so
short a time God has forgiven me all my sins, which are heavier
than the sand of the sea'.
The angel said to him: 'I did not spare
Zacharaiah the priest when he refused to believe me concerning his
son.
<Lk 1:20> I imprisoned his tongue to teach him not to doubt
what I said; how then should I now spare you? You shall be totally
incapable of speech from this time onwards.' Abba David prostrated
himself before the angel and said: 'When I was in the world,
committing abominable acts and shedding blood, I had the gift of
speech.
Will you deprive me of it by imprisoning my tongue, now
that I wish to serve God and offer up hymns to him?' The angel
replied: “You will only be able to speak during the services.
At all
other times you shall be completely silent'—and that is how it was.
He sang the psalms, but he could say no other word, big or little.
The one who told us these things said: 'I saw him many times and
I glorified God'.

