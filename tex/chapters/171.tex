171, THE LIFE OF TWO REMARKABLE MEN,
THEODORE THE PHILOSOPHER
AND ZOILOS THE READER

In Alexandria there were two wondrously virtuous men, Abba
Theodore the philosopher and Zollos the reader; we were well
acquainted with both of them, the one from his lectures, the other

because we shared the same homeland and up-bringing- Abba
Theodore had no possessions whatsoever, except for a philosopher\textquotesingle s
cloak and a few books. He slept on a bench whenever he came
across a church. He finally renounced the world at the Community
of Salama and there he ended his days. As for Zoilos the reader, he
too was equally indifferent to possessions. He too had nothing but
a philosopher\textquotesingle s cloak, a very old suit and a few books. Calligraphy
was his occupation. When he died in the Lord, he was buried at
Lithazomenos, in the monastery of Abba Palladios.

Some fathers went to Master Cosmas the lawyer and asked him
about Abba Theodore the philosopher and Zoilos the reader: which
of the two had progressed farther in the practice of asceticism? He
replied: 'They both had the same kind of clothing, the same kind of
bed and the same kind of food. They both rejected anything that
was in excess of basic necessities. They were equal in humility,
poverty and self-discipline. But Abba Theodore, who went bare-foot
and suffered greatly with his eyes, had learnt both the Old and the
New Testaments by heart. But he had also the consolation of the
company of the brethren, contact with friends; a not inconsiderable
distraction when he was active and when he was teaching. In the
case of Zoilos the reader, not only is his isolation from the world
<xeniteia> praiseworthy, but so is his solitude <erémia>, his
immense toil, the way he keeps his tongue on a leash. He had no
friend, nothing to cal] his own, no-one to talk to; he engaged in no
worldly activity; allowed himself no relief, nor would he accept the
smallest service from anybody. He did his own cooking and
washing; he allowed himself no pleasure from reading. He was ready
to be of service to others; neither cold nor heat nor bodily sickness
was of any account with him. He shunned laughter, sadness,
inactivity, relaxation. For all the exiguity of his clothing he was
constantly devoured by lice. Yet this man, in comparison with the
first one, had no small consolation from his freedom of movement,
for he was entirely free to go wherever he would, by night or by

day. Yet even this freedom was of no avail because so heavy was his
toil, that he never made use of it. Thus he appeared studiously to
avoid over-much contact with the world, Each one of them shall
receive his own reward, consonant with his toil, and the progress he
made; with his spiritual and mental purity, his fear of God, his
charity, his worship, his compunction <katanu.xis>, his constancy
in psalm-singing and prayer, his persistent faith—and the good
pleasure of God, which is hidden and concealed from people.

