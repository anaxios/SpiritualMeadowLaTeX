145, THE LIFE OF THE BLESSED GENNADIOS,
PATRIARCH OF CONSTANTINOPLE,
AND OF His READER, CHARISIOS

We visited the Community of Salama, <nine miles out of> Alexan-
dria, and there we found two elders who said they were priests of
the church of Constantinople.
[They spoke to us about Gennadios,
the blessed Patriarch of Constantinople saying, that he was very
gentle, pure of body and very much in contro! of himself.] And they
told us this about him: that he was troubled by many <people
complaining> about a cleric who was leading a very dissolute life,
a man named Charisios.
The patriarch sent for him and tried to
correct him by exhortation, but when nothing was achieved by this,
he proceeded to chastise and discipline him after the manner of a
father and a churchman.
The patriarch realised that this was doing
the cleric no good, for now he was indulging in murder and
dabbling in witchcraft.
So he sent one of the agents in his service,
ordering him to say to the holy martyr Eleutherios (in whose
oratory Charisios served as lector): 'Saint Eleutherios, your officer
is a great sinner.
Either reform him, or get rid of him.' So the agent
came to the oratory of the holy martyr Eleutherios and, standing
before the altar, turning towards the apse, he stretched out his hand
and said to the martyr: 'Holy martyr of Christ, the Patriarch
Gennadios declares to you, through me, sinner though I be, that
your officer is deeply in sin.
You are either to reform him or get rid

of him.' Next day, that worker of evil deeds was found dead.
All
were amazed, and glorified God.

