84, THE LIFE AND DEATH OF AN ANCHORITE
OF THE SAME MONASTERY, A SERVANT OF GOD

The fathers of the same monastery told us this:

There was an anchorite in these mountains, a great man in the eyes
of God who survived for many years on the natural vegetation
which could be found there.
He died in a certain small cave and we
did no know, for we imagined that he had gone away to another
wilderness place, One night this anchorite appeared to our present
father, that good and gentle shepherd, Abba Julian, as he slept,
saying to him: “Take some men and go, take me up from the place
where I am lying, up on the mountain called The Deer'.
So our
father took some <brethren> and went up into the mountain of
which he had spoken.
We sought for many hours but we did not

come across the remains of the anchorite.
With the passage of time,
the entrance to the cave <in which he lay> had been covered over
by shrubs and snow.
As we found nothing, the abba said: 'Come,
children, let us go down',—and just as we were about to return, a
deer approached and came to a standstill some little distance from
us.
She began to dig in the earth with her hooves.
When our father
saw this, he said to us: 'Believe me, children, that is where the
servant of God is buried'.
We dug there and found his relics intact.
We carried him to the monastery and buried him <there>.

