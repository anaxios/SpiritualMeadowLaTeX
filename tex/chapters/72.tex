72, ABBA PALLADIOS' STORY OF AN OLD MAN WHO
COMMITTED MURDER AND FALSELY ACCUSED
A YOUTH OF THE SAME CRIME

Abba Palladios told us that an old fellow living in the world was
arrested for murder. When he was tortured by the magistrate of
Alexandria he said that somebody else had been involved in the
murder as his accomplice, a young fellow about twenty years old.
They were both severely tortured. The old fellow said: “You were

with me when I committed the murder'. The youth denied having
anything to do with the affair, nor had he been with the old fellow.
When they had both been severely tortured they were condemned
to be hanged. So they went out to the fifth <mile-post from the
city> to where it is customary to punish such criminals. About one
stade away there is a ruined temple of Kronos. When they came to
the place, the populace and the soldiers wanted to hang the youth
on the scaffold first. He made a profound act of obeisance before
the soldiers and said: 'For the sake of the Lord, of your charity
hang me towards the east so that I may look in that direction when
I am hanging there alone'. The soldiers said to him: 'Why so”? He
replied: 'In truth, sirs, it is only seven months since your unworthy
<servant> received baptism and became a Christian'. When they
heard this, the soldiers wept over the youth. The old fellow called
out in great anger: “By Serapis, hang me so I look towards Kronos!'
When the soldiers heard the blasphemy of the old fellow, they left
the youth aside and hung the old one. And as they were doing this,
a mounted messenger arrived from the prefect and said to the
soldiers: 'Do not execute the youth, bring him back'. This brought
joy to all the soldiers who were there. They took him and brought
him back into the praetorium, and the prefect released him. Having
been rescued when he despaired <of rescue>, the young man went
and became a monk. We have written this for the benefit of the
many [and of ourselves, so we might be aware that the Lord knows
how to deliver the godly from temptation).

