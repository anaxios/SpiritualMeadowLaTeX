216. SOME GOOD ADVICE
ABOUT NEITHER BEING OBDURATE
NOR REMAINING OBDURATE

Once when J was in the Holy City a person who loved Christ came
to me and said: 'There had been a small altercation between my
brother and me and he will not be reconciled with me. You go
speak to him and reason with him', I received this commission
joyfully. I called the brother and spoke to him of those things which
tend to love and peace, and it seemed as though he was coming
\_ Found to my point of view. At last, he said to me: 'I cannot be
reconciled with him because I swore on the cross'. I said to him
with a smile: 'Your oath was equivalent to saying: “Oh Christ, by

the honourable cross, I will not keep your commandments, but I
will do the will of your enemy the devil”. We ought not only to put
a halt-to what we have set in motion, but also (and even more so)
to repent and lament for what we have wrongly instigated to our
own hurt. As the divinely inspired Basi] says: “If Herod had
repented and not kept his oath, he would not have committed that
heinous sin of beheading John the Forerunner of Christ”'. Finally
I brought out the opinion of Saint Basil which he took from the
gospel: that when Christ wanted to wash the feet of Saint Peter,
although <the apostle> obstinately refused at first, he afterwards
changed his mind, <Some manuscripts add: when he heard this, he
was reconciled with his brother>.

