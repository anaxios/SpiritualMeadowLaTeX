172.
THE LIFE OF THE ABOVE-MENTIONED
COSMAS, THE LAWYER

Concerning this master Cosmas the lawyer, many people told us
many things; some one thing, others another.
But most people told
us a great deal.
We shall write down what we saw with our own
eyes and what we have carefully examined, for the benefit of those
who chance to read it.
He was a humble man, merciful, continent,
a virgin, serene, cool-tempered, friendly, hospitable, and kind to the
poor.
This wondrous man greatly benefitted us, not only by letting
us see him and by teaching us, but also because he had more books
than anybody else in Alexandria and would willingly supply them
to those who wished.
Yet he was a man of no possessions.
Through-
out his house there was nothing to be seen but books, a bed and a
table.
Any man could go in and ask for what would benefit
him—and read it.
Each day I would go in to him and I never
entered without finding him either reading or writing against the
Jews.
It was his fervent desire to convert the Hebrews to the truth.
For this reason he would often send me to some Hebrews to discuss
some point of Scripture with them, for he would not readily leave
the house himself.

One day I went to the house of Master Cosmas the lawyer and,
as I was quite familiar with him, I said to him: 'Of your charity,
how long have you been leading the solitary life? <hésuchazén>'He

kept his silence and gave no answer, so I asked again: 'For the sake
of the Lord, tell me'.
He remained silent a little longer, then he told
me: 'For thirty-three years', When I heard this, I glorified God.
Another time I came to him and asked him: 'Of your extreme
charity, and in full knowledge that it is for the benefit of my soul
that I ask you this; will you tell me what you have accomplished in
so long a period of solitude and continence <hésuchia...egkrateia>Y
He heaved a great sigh from the depths of his heart and said to me:
"What shall a man Jiving in the world accomplish, especially a man
who stays in his own house?' Yet I begged him to tell me; for the
Lord's sake, and for the good of my soul, Finally, coerced by my
persistence, he said: 'Forgive me; there are three things I know of
which I have accomplished: not to laugh, not, to swear and not to

lie'.

