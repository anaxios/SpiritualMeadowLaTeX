176. THE BEAUTIFUL STORY OF ABBA ANDREW
ABOUT TEN TRAVELLERS,
OF WHOM ONE WAS A HEBREW

Abba Palladios told us he had heard one of the fathers whose name
was Andrew (whom we also met) say:

When we were in Ajexandria, Abba Andrew at the eighteenth <mile
post> told us, saying:*

As a young man I was very undisciplined. A war broke out and
confusion reigned so, together with nine others, I fled to Palestine.
One of the nine was a fellow with initiative and another was a
Hebrew. When we came into the wilderness, the Hebrew became
mortally sick, so we were in great distress, for we did not know
what to do for him. But we did not abandon him. Each of us
carried him as far as he was able. We wanted to get him to a city
or to a market-town so that he should not die in the wildemess. But
when the young man was completely worn out and was brought to
the point of death by hunger and a burning fever, by utter exhaus-
tion and a raging thirst from the heat (in fact he was about to
expire), he could no longer bear to be carried. With many tears, we
decided to abandon him in the wilderness and go our way. We

could see death from thirst lying in store for us. We were in tears
when we set him down on the sand. When he saw that we were
goilig to leave him, he began to adjure us, saying: 'By the God who
is going to judge both the quick and the dead, leave me to die not
as a Jew, but as a Christian. Have mercy on me and baptise me so
that I too may depart this life as a Christian and go to the Lord.'
We said to him: 'Truly, brother, it is impossible for us to do
anything of the sort. We are laymen and baptising is bishops' work
and ptiests'. Besides, there is no water here.' But he continued to
adjure us in the same terms and with tears, saying: 'Oh, Christians,
please do not deprive me of this benefit'. While we were most
unsure what to do next, the fellow with initiative* among us,
inspired by God, said to us: 'Stand him up and take off his clothes'.
We got him to his feet with great difficulty and stripped him. The
one with initiative filled both his hands with sand and poured it
three times over the sick man's head saying: 'Theodore is baptised
in the name of the Father and of the Son and of the Holy Spirit',
and we all answered amen to each of the names of the holy,
consubstantial and worshipful Trinity. The Lord is my witness,
brethren, that Christ, the Son of the living God, thus cured and re-
invigorated him so that not a trace of illness remained in him. In
health and vigour he ran before us during the rest of our journey
through the wilderness. When we observed so great and so sudden
a transformation, we all praised and glorified the ineffable majesty
and loving kindness of Christ our God. When we arrived at
Ascalon, we took this matter to the blessed and saintly Dionysios,
who was bishop there, and told him what had happened to the
brother on the journey. When the truly holy Dionysios heard of
these things, he was stupefied by so extraordinary a miracle, He
assembled all the clergy and put to them the question of whether he
should reckon the effusion of sand as baptism or not. Some said
that, in view of the extraordinary miracle, he should allow it as a
valid baptism; others said he should not. Gregory the Theologian

enumerates all the kinds of baptism.* He speaks of the Mosaic
baptism, baptism in water, that is, but before that of baptism ina
cloud and in the sea. 'The baptism of John was no longer Judaic
baptism, for it was not only a baptism in water, but also unto
repentance. Jesus also baptised, but in the Spirit, and this is
perfection. I know also a fourth baptism: that of martyrdom and of
blood. And I know a fifth: the baptism of tears.' 'Which of these
baptisms did he undergo', asked some, 'so that\textquotesingle we might pronounce
on its validity? For indeed the Lord said to Nicodemos: Except a
man be born again of water and of the spirit, he shall not enter into
the kingdom of heaven <Jn 3;5.>' Others objected to this: 'How so?
Since it is not written concerning the apostles that they were
baptised, shall they not enter the kingdom of heaven?' To this,
others replied: 'But indeed they were baptised, as Clement, the
author of Stromatés, testifies in the fifth book of Hypotyposes. * In
commenting on the saying of the Apostle Pau!, he opines: J thank
God that I baptised none of you <1 Co 1:14> that Jesus is said to
have baptised none but Peter; Peter to have baptised Andrew;
Andrew, James and John, and they the others'. When they had said
all this and much more beside, it seemed good to the blessed Bishop
Dionysios to send the brother to the holy Jordan and for him to be
baptised there. The fellow with initiative he ordained deacon.

