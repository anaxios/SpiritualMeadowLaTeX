127, THE LIFE OF ABBA GEORGE
OF THE HOLY MOUNTAIN
OF SINAI AND OF ANOTHER PERSON,
ONE FROM PHRYGIAN GALATIA

This story was told to us by Amma Damiana the solitary, the
mother of Abba Athenogenes, Bishop of Petra:

There was a higoumen at Mount Sinai who was truly great, and an
ascetic, George by name. As he was sitting in his cell one Holy
Saturday, this Abba George conceived a desire to celebrate the holy
resurrection in the Holy City and to partake of the holy mysteries
in the Church of the Holy Resurrection of Christ our God. All day
long the elder continued in prayer meditating upon the validity of
these thoughts. With evening, his disciple came and said: 'Father,
give the word for us to proceed to the canonical service'. The elder
replied: 'You go, and when it is time for holy communion, return
<home and I will> come too. Then the elder stayed in his cell.
When it came to the time for holy communion at the <Church of
the> Holy Resurrection, the elder was found near the blessed
Bishop Peter and he, together with the priests, was given commu-
nion by <the bishop>. When the patriarch saw him, he said to his
syncellos,* Menas: 'When did the abbot <abbas>* of Sinai come
here'? The syncellos replied: 'With all due respect, my lord, I had
not seen him until only this very instant'. Then the patriarch said to
the syncellos: 'Tell him not to go away; I want him to take food
with me', The syncellos went and said this to the elder, who
responded: 'The will of God be done'. When the elder had left the
service and venerated the holy sepulchre, he found himself back in
his cell again, and there was his disciple knocking at the door and
saying: 'Father, if you please, come and communicate'. The elder
went to the church with his disciple and partook of the holy
mysteries. Archbishop Peter was saddened that <the elder> should
have disobeyed him. After the feast, he sent him a letter; likewise to

Abba Photios, Bishop of Paran, and to the father of Sinai, telling
them to bring the abba to him. When the carrier of the letters
arrived and had delivered them, <the abba> sent three priests to the
patriarch: Abba Stephan the Cappadocian, 'the great'; Abba
Zosimos of whom we have spoken above; and Abba Dulcitius, a
Roman. The elder sought to justify himself by writing: 'My most
holy lord; God forbid that I should disregard your holy messenger'.
Then he wrote this: 'I would have your blessedness know that, six
months from now, we are going to meet each other in the presence
of the Lord Christ our God; and there, I will make an act of
obeisance to you'. The priests went their way and gave the letter to
the patriarch. They said it was many years since the elder had come
up to Palestine. They showed him a letter from the Bishop of Paran
certifying that for about seventy years the elder had not departed
from the holy Mount Sinai. The godly and gentle Peter <accepted
as> witnesses the bishops who were there and the clergy, who said:
'We saw the elder and we all greeted him with a holy kiss'. Six
months later, both the elder and the patriarch died, as the elder had
foretold.
The same Amma Damiana told us this too:

On a <Good-> Friday before I was enclosed, I went to <the Church
of> Saints Cosmas and Damian and spent the whole night there. In
the evening, there came an old woman, a native of Phrygian
Galatia, and she gave two lepta to everybody who was in the
church. I knew her because she had often given me <alms>. One
day a kinswoman of mine (and of the most faithful Emperor
Maurice* came to pray at the Holy City and stayed there for a year.
Taking her with me, I went to Saints Cosmas and Damian. While
we were in the oratory, I said to my kinswoman: 'Look, my lady;
when an old woman comes distributing two coins to each person,
please swallow your pride and accept them'. With obvious distaste,
she said: 'Do I have to accept them?' 'Yes', I said: 'Take them, for
the woman is great in the eyes of God. [She fasts all week long; and

whatever she is able to gain by this discipline] she distributes it
among those who are found in the church. She is a widow of about
eighty years of age; take <the coins> she offers you and give them
to somebody else, Do not refuse the sacrifice of this old woman.' As
we were speaking in this way, the old woman came in and began
her almsgiving. In silence ard with serenity she came and gave me
<some coins>. She gave some to my kinswoman too, saying: 'Take
these, and eat'. When she had gone, we realised that God had
revealed to her that I had said: 'Take them and give them a poor
person'. <My kinswoman> therefore sent a servant of hers to get
vegetables with the two coins. These she ate, and she affirmed
before God that they were as sweet as honey. This both astonished
her, and led her to give thanks to God who endows his servants
with grace.

