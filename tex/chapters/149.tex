149. THE AMAZING TALE OF AMOS,
PATRIARCH OF JERUSALEM
CONCERNING THE MOST SACRED LEO,
THE ROMAN PONTIFF

When Abba Amos went down to Jerusalem and was consecrated
patriarch, all the higoumens of all the monasteries went up to do
homage to him and, amongst them, I also went up, together with
my higoumen. The patriarch started saying to the fathers: 'Pray for
me, fathers, for I have been handed a great and difficult burden and
I am more than a little terrified at the prospect of the patriarchal
office. Peter and Paul and Moses, men of their stature are adequate
shepherds of the rational sheep, but I am a person of little worth.
Most of all, I fear the burden of ordinations.* I have found it
written that the blessed Leo who became primate of the church of
the Romans, remained at the tomb of the Apostle Peter for forty
days, exercising himself in fasting and prayer, invoking the Apostle
Peter to intercede with God for him, that his faults might be
pardoned. When forty days were fulfilled, the apostle appeared to
him, saying: 'T prayed for you, and all your sins are forgiven, except
for those of ordinations. This alone will be asked of you: whether
you did well, or not, in ordaining those whom you ordained'.

