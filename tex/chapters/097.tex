97.
THE LIFE AND DEATH OF TWO BROTHERS
WHO SWORE NEVER TO BE SEPARATED FROM EACH OTHER

Abba John the anchorite, 'John the Red' as he was called, said: I
have heard Abba Stephan the Moabite say that when he was in the
Community of Saint Theodosios, the great superior of the commun-
ity, two brothers were there who had sworn an oath to each other
that they would never be separated from each other, either in life or
in death.
Whilst they were in the community and a source of
edification for all, one of the brothers was attacked by a yearning
for fornication, Unable to withstand this attack, he said to his
brother: 'Release me, brother, for I am driven towards fornication
and I want to go back to the world'.
The other brother began to
beg and entreat him, saying: 'Oh, brother, do not destroy all you
have endured', He replied: 'Either come with me so that I can do
the deed, or release me to go my own way'.
The brother did not
want to release him—so he went into the city with him.
The
afflicted brother went into the house of fornication whilst the other
brother stood outside.
Taking up dust from the ground he threw it
on his own head, reproaching himself.
When the brother who had
gone into the brothel came out again, having done the deed, the
other brother said to him: 'My brother, what have you gained by
this sin, and what have you not lost by it? Let us go back to our
place'.
The other replied: 'I cannot go back into the wildemess
again.
You go: I am staying in the world'.
When the first brother
had done all he could and still failed to persuade the other to follow
him into the wilderness, he too remained in the world with his
brother.
They both worked as labourers to support themselves.

It was about this time that Abba Abraham (who had already
founded the so-called 'Monastery of the Abrahamites' at

Constantinople, he who later became Archbishop of Ephesos, a
good and gentle shepherd), it was about this same that he built his
own monastery, the one known as 'The Monastery of the Byzan-
tines' <at Olivet, west of Jerusalem>.
The two brothers came there
and worked as labourers, for which they received wages.
The one
who had fallen prey to fornication would take both their wages and
go off to the city each week where he would squander <their
earnings> in riotous living.
The other brother would fast all day
long, performing his work in profound silence, not speaking to
anybody.
When the workmen noticed that he neither ate nor spoke
each day but was always deep in thought, they told the saintly
Abraham about him and his way of life.
Then the great Abraham
summoned the workman to his cell and asked him: “Where are you
from, brother, and what kind of work so you do?' The brother
confessed all to him, 'It is because of my brother that I put up with
all this, in the hope that God will look upon my affliction and save
my brother'.
When the godly Abraham heard this, he said to the
brother: 'The Lord has granted you the soul of your brother too'.
Abba Abraham dismissed the brother, who left his cell and, behold!
there was his brother, crying: 'My brother, take me into the
wilderness so I can be saved'.
He immediately took him and went
to a cave near to the holy Jordan where he locked him up [in which
they shut themselves}.
After a little while, the sinful brother, having
made great spiritual progress in the things that are God's, departed
this life.
The other brother, faithful to the oath, remained in the
cave and eventually he too died there.

