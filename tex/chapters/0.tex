JOHN EVIRATUS

TO HIS BELOVED IN CHRIST,
SOPHRONIOS THE SOPHIST

In my opinion, the meadows in spring present a particularly delightful prospect.
They display to the beholder a rich diversity of flowers which arrests him with its charm, for it brings delight to his eyes and perfume to his nostrils.
One part of this meadow blushes with roses; in another place lilies predominate, drawing one's attention to themselves and away from the roses.
In another part the colour of violets blazes out, resembling the imperial purple.
In short, the diversity and variety of innumerable flowers affords
delights both to nostril and to eye on every side.

Think of this present work in the same way Sophronios, my sacred and faithful child.
For in it, you will discover the virtues of holy men who have distinguished themselves in our own times;
men, as the Psalmist says, planted by the waterside <Ps 1:3>.
They were all equally beloved of God (by the grace of Christ),--yet there was a diversity in the virtues from which the beauty and the charm of each derived.
From among these I have plucked the finest flowers of the unmown meadow and worked them into a crown which I now offer to you, most faithful child; and through you, to the world at large.

I have called this work meadow on account of the delight, the fragrance and the benefit which it will afford those who come across
it.
For the virtuous life and habitual piety do not merely consist of
studying divinity; not only of thinking on an elevated plain about
things as they are here and now.
It must also include the description in writing of the way of life of others.
So I have striven to complete this composition to inform your love, oh child;
and as I have put together a copious and accurate collection, so I have emulated the most wise bee, gathering up the spiritually beneficial deeds of the fathers.
Now I will begin to tell <you> those things.
