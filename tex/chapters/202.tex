202.
THE LIFE OF THE SERVANT OF GOD,
ABIBAS, THE SON OF A WORLDLY MAN,

One of the fathers said there was a man living in the world who had
a pious son, pure and temperate in all things, who, from his
childhood, had not drunk wine.
It was his intention to withdraw
from the world.
The father wanted him to become involved in
business matters but the son was reluctant.
There were other
brothers.
but he was the oldest.
As his father\textquotesingle s wishes and his own
could not be reconciled, the father was always reproaching him and
casting his temperance in his teeth, saying: 'Why are you not like
your brothers, and why do you not get yourself involved in business
affairs?' The son endured it all in silence; everybody loved him for
his piety and his moderation <sdéphrosuné>.

When the father was dying, some of the family, together with
others who friends of Abibas, for that was the son's name, came
together and said: 'Perhaps the father will deny the servant of God
his inheritance', for they thought that he hated his son from the way
he used to revile him.
They resolved to intercede with the father
(who was sick) on his son's behalf.
They went to him and said: 'We
have a favour to ask of you'.
He said to them: 'What would you ask
of me"? They said: 'It concerns Master Abibas.
We want to ask you
not to despise him'.
He said: 'You want to ask a favour of me for
him? They said they did, and he continued: 'Call him here to me'.
They thought he was going to reproach him as usual.
When the son
came in, the father told him to come near <to him>—which he did.
And then the father collapsed in tears at his feet, saying: 'Forgive
me, my child, and pray to God that the wrong I have done you be
not be held against me.
For you were seeking for Christ and I was
burying myself with worldly affairs.' He called his other sons and
said to them: 'This is your master and your father.
Whatever he
says you may have, that you may have; and whatever he says you
may not have, that you may not have'.
They were all astonished.
The father then died.
<Abibas> gave to each brother his share of
the inheritance and he took his own share too, but he gave it all to
the poor, leaving nothing for himself.
He built a small cell into
which he could withdraw from the world and when the cell was
completed, he fell ill.
His end was approaching.
His <monastic>
brother was sitting with him, to whom the dying man said: 'Go and
keep company with your household, for it is a holy day' (it was the
feast of the Holy Apostles).
The brother replied: 'How could I go
and leave you?' The other replied: 'Go; and when the time comes,
I shall call you'.
When the time came, he stood at the window and
knocked.
The brother heard and obeyed the sick man's signal to
come, As soon as he entered, the older brother surrendered his soul
to the Lord.
Everybody was amazed and glorified God, saying: 'His
end was worthy of the love with which he loved Christ'.

