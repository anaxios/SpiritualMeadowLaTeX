150, THE LIFE AND HOLINESS
OF THE BISHOP OF ROMILLA*

Abba Theodore told us that thirty miles from Rome there is a small
town called Romilla. In that town there was a very great and
virtuous bishop. One day some of the people of Romilla came in to
the most blessed Agapetos, Pope of Rome, and made charges
against their own bishop to the pope, saying that he ate from a
consecrated paten. The pope was shocked when he heard this. He
sent two clerics to bring the bishop, bound, to Rome, on foot; and

he threw him into prison when he arrived. When the bishop had
been in prison three days, Sunday came around. Whilst the pope
was sleeping, as dawn broke on the Sunday morning, he saw-in his
sleep one who <stood beside him and> said: 'You are not to
celebrate the eucharist this Sunday, neither you nor any other of the
clergy and bishops who are in this city, except the bishop whom you
are holding in prison. I want him to celebrate the eucharist this day.
When the pope awoke, he said to himself concerning the vision he
had seen: 'I have received such a complaint against him, and Ae is
to celebrate the eucharist?' A second time the voice came to him in
his sleep, saying: 'I told you: that bishop; who is in prison, 4e shall
celebrate the eucharist', Likewise a third time the figure appeared to
him as he was grappling with the problem and said the same thing
to him. When the pope awoke, he sent to the prison and had the
bishop brought out. Then he questioned him: 'What is your way of
life <ergasia>? But the bishop would answer nothing other than: 'I
am a sinner'. As he could not persuade the bishop to say anything
else, he said to him: 'Today you shall celebrate the eucharist'. When
he stood at the holy altar with the pope beside him and the deacons
in a circle around the altar, the bishop began the prayer of conse-
eration; but before adding the conclusion, he began the prayer of
consecration al] over again for a second, a third, and a fourth time.
Everybody was astonished at such repetition and the pope said to
him: 'What is this then, that you are starting the holy prayer for a
fourth time and do not bring it to a conclusion?' Then the bishop
replied: 'Forgive me, holy pope, but I do not perceive the coming
of the Holy Ghost <epiphoitésis> as is usually the case; that is why
I do not conclude <the prayer>. However, my sacred lord, would
you send that deacon holding the fan <rhipidion>* away from the
altar, for I do not dare to tell him to go.' Then the godly Agapetos
gave the order and the deacon went away. Straightaway the bishop
and the pope saw the presence <parousia> of the Holy Ghost; but
the curtain which was above the altar moved of its own volition and

overshadowed the pope, the bishop, all the deacons who were in
attendance and even the holy altar itself, for three hours. Then the
godly Agapetos realised that this was a great bishop who had been
falsely accused. And so great was his distress at having wronged him
that he resolved never again to make any hasty decision, but to act
with much thought and great patience.*

