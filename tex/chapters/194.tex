194.
THE EXHORTATION OF AN ELDER
WHO LIVED AT SCETE TO A MONK,
NOT TO ENTER TAVERNS

The was a monk living at Scété who went up to Alexandria to sell
his handiwork and he saw a younger monk go into a tavern.
This
troubled the elder, so he waited outside, intending to meet the monk
when he came out, which is indeed what happened.
When the
younger monk came out, the elder took him by the hand and led
him aside, saying to him: 'Brother, do you not realise that you are
wearing the holy habit, sir? Do you not know that you are a young
man? Are you not aware that the snares of the devil are many? Do
you not know that monks who live in cities are wounded by means
of their eyes, their hearing and their clothing? You went into the
tavern of your own free will; you hear things you do not want to
hear and see things you would rather not see, dishonourably
mingling with both men and women.
Please do not do it, but flee to
the wilderness where you can find the salvation you desire.' The
young man answered him; 'Away with you, good elder.
God
requires nothing but a pure heart.' Then the elder raised his hands
to heaven and said: 'Glory to you, Oh God, that I have spent fifty
years at Scété and have not acquired a pure heart, yet this man,
who frequents taverns, has attained pureness of heart'.
He turned to
the brother and said: 'May God save you and sot disappoint me in
my hope' <Ps 118:116>.

