178.
THE LIFE OF AN ELDER OF THE
COMMUNITY OF THE SCOLARII, A SIMPLE MAN

Abba Gregory, priest of the Community of the Scholarii, told us
that at Monidia there lived a monk who was an exceedingly hard
worker, but somewhat indiscriminating in matters of faith.
He
would receive holy communion indiscriminately, in whatever church
he happened to be, One day an angel of God appeared to him and
said: 'Tell me elder, when you die, how do you want us to bury
you? The way the Egyptian monks bury <the dead>, or after the
custom of Jerusalem?' The elder said he did not know, and the
angel replied: 'Think about it.
I will come to you three weeks from
now and you shall tel] me.' The elder went to a colleague and told
him what he had heard from the angel.
The <second> elder was
utterly amazed at what he heard.
He stared at the man for a long
time; then, inspired by God, he said to him: 'Where do you partake
of the holy mysteries” The other replied: 'Wherever I happen to be'.
The elder said to him: 'Never again should you communicate
outside the holy catholic and apostolic Church in which the four

 

holy councils are named: the council of the three hundred and
eighteen fathers at Nicaea, that of the one hundred and fifty fathers
at Constantinople, the first Council of Ephesos of two hundred, and
that of the six hundred and thirty fathers at Chalcedon.
And when
the angel comes, say to ὁ μῖπε “T wish to be buried according to the
custom of Jerusalem.”

Three weeks Jater, the angel came sind said to him: 'Which is it
to be, elder? Have you given thought to the matter?” The elder
replied that he wished to be buried according to the Jerusalem
custom.
The angel replied: 'Very well, very well', and the elder
immediately surrendered his soul.
This was done so that the elder
would not lose his labour and be condemned as a heretic.

