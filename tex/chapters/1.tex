\chapter{THE LIFE OF JOHN THE ELDER AND THE CAVE OF SAPSAS}

There was an elder living in the monastery of Abba Eustorgios*
whom our saintly Archbishop of Jerusalem wanted to appoint
higoumen of the monastery.
<The candidate> however would not agree and said: 'I prefer praye on Mount Sinai', The archbishop* urged him first to become <higoumen> and then to depart <for the mountain> but the elder would not be persuaded.
So <the arch-bishop> gave him leave of absence, charging him to accept the office of higoumen on his return.
<The elder> bid the archbishop farewell and set out on the journe to Mount Sinal, taking his own disciple* with him.
They crossed the river Jordan* but before they reached even the first mile-post the elder began to shiver with fever.
As he was unable to walk, they found a small cave and went into it so that the elder could rest.
He stayed in the cave for three days, scarcely able to move and burning with fever.
Then, whilst he was sleeping, he saw a figure who said to him: 'Tell me, elder, where do you want to go'?
He replied: 'To Mount Sinai'.
The vision then said to him: 'Please, I beg of you, do not go there', but as he could not prevail upon the elder, he withdrew from him.
Now the elder's fever attacked him more violently.
Again the following night the same figure with the same appearance came to him and said: `Why do you insist on suffering like this, good elder?
Listen to me and do not go there,' The elder asked him: 'Who then are you'?
The vision replied: 'I am John the Baptist and that is why I say to you: do not go there.
For this little cave is greater than Mount Sinai.
Many times did our Lord Jesus Christ come in here to visit me.
Give me your word that you will stay here and I will give you back your health'.
The elder accepted this with joy and gave his solemn word that he would remain in the cave.
He was instantly restored to health and stayed there for the rest of his life.
He made the cave into a church and gathered a brotherhood together there;
the place is called Sapsas.*
Close by it and to the left is the Wadi Chorath* to which Elijah the Tishbite was sent during a drought, it faces the Jordan.

