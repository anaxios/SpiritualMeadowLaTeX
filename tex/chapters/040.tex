40, THE LIFE OF ABBA COSMAS THE EUNUCH

This story was told to us by Abba Basil, priest of the monastery of
the Byzantines:*

When I was with Abba Gregory the Patriarch at Theoupolis,* Abba
Cosmas the Eunuch of the Lavra of Pharén came from Jerusalem.
This man was most truly a monk, orthodox and of great zeal, with
no small knowledge of the holy Scriptures. After being there a few
days, the elder died. Wishing to honour his remains, the patriarch
ordered that he should be buried at a spot in the cemetery where a
bishop lay. Two days later I came to kiss the elder's grave. A poor
man stricken with paralysis was lying on top of the tomb, begging
alms of those who came into the church. When this poor man saw
me making three prostrations and offering the priestly prayer, he

said to me: 'Oh abba, this was indeed a great elder, sir, whom you
buried here three days ago'. I answered him saying: 'How do you
know that'? He told me: 'I was paralysed for twelve years and,
through this elder, the Lord cured me. When I am distressed, he
comes and comforts me, granting me relief. And now you are about
to hear yet another strange thing about this elder. Ever since you
buried him, I hear him at night calling and saying to the bishop:
“Touch me not; stay away! Come not near, thou heretic and enemy
of the truth and of the holy catholic Church of God'. Having
heard this from the man cured of his paralysis, I went and Tepeated
it to the patriarch. I besought that most holy man to let us take the
body of the elder and lay it in another tomb. Then the patriarch
said to me: 'Believe me, my child, Abba Cosmas will suffer no hurt
from the heretic. This has all come about that the virtue and zeal of
the elder might become known to us after his departure from this
world; also that the doctrine of the bishop should be revealed to us,
so that we not hold him to have been one of the orthodox'.

The same Abba Basil also told us this concerning this elder,
Abba Cosmas:
I visited him when he was staying at the Lavra of Pharén and he
said to me: A doubt once perplexed me concerning the saying of the
Lord to his disciples: “He who has a garment, let him sell it and buy
a sword', and they said to him: 'Here are two swords'.* After
agonizing unsuccessfully over the meaning of this passage, I went
from my cell, out into the heat of the midday sun, driven by a
compulsion to go to the Lavra of Pyrgia <=The Towers or Turrets>
where Abba Theophilos was, to ask him about the matter. When I
came into the desert, near to Calamén, I saw an exceedingly large
dragon coming down from the mountain towards Calam6n. It was
so large that it made a great vault of itself as it moved. I suddenly
realised that I was passing through its vault unharmed. I knew (he
said) that the devil was trying to frustrate my purpose but that the
prayer of the elder had prevailed. I went my way (he said) and

recited the passage of scripture to Abba Theophilos. He told me the
explanation of the two swords® is this: the active and the contem-
plative [τὸ πρακτικὸν καὶ τὸ θεωρητικόν). ΠῚ a person
has these two virtues, he is approaching perfection. I visited this
Abba Cosmas at the Lavra of Pharén and stayed there for ten
years, Whilst he was speaking to me about the salvation of the soul,
we came across an opinion of Saint Athanasios, Archbishop of
Alexandria. The elder said to me: 'When you come across a saying
of Athansios the Great, if you have no paper, write it on your
clothing'—so great was the appetite of this elder for our holy fathers
and teachers. They also said this about him: that on the eve of the
holy Lord\textquotesingle s Day, he would stand from vespers to dawn singing and
reading, in his cell or in church, never sitting down at all. Once the
sun had risen and the appointed service had been sung, he would sit
reading the holy Gospel until it was time for the eucharist.

