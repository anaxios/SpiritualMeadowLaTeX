49. THE WONDROUS VISION OF THE DUKE OF
PALESTINE BY WHICH HE WAS COMPELLED
TO RENOUNCE THE AFOREMENTIONED HERESY
AND TO ENTER INTO COMMUNION
WITH THE CHURCH OF CHRIST

Anastasios the Priest also told us that when Gébémer became the
military governor of Palestine, his first act was to come and worship

at the holy <Church of the> Resurrection of Christ who is God. As
he was about to approach, he saw a ram charging at him intent on
impaling him on its horns, So great was his fear that he stepped
backwards towards the guardian of <the Chapel of> the Cross who
was present, and also the lictors who stood by. They said to him:
'What is the matter, your highness? Why do you not enter”? He
said: 'Why did you bring in that ram”? They were taken aback by
this, but they peered into the holy sepulchre and saw nothing. So
they spoke to him, urging him to enter and telling him that there
was no such thing <as a ram> in there. A second time he made as
though to enter and again he saw the ram charging at him and
preventing him from entering. This happened several times, at least
in Ais eyes. Those who were with him saw nothing and the guardian
of <the Chapel of> the Cross said to him: 'Believe me, your
highness, there is something in your soul and it is because of this
that you are prevented from worshipping at the holy and life-giving
sepulchre of our Saviour. You would do well to confess before God,
for he is kindly disposed towards humanity and it was to show
mercy on you that he made you see this vision'. Bursting into tears,
the governor said: 'I am responsible for many great sins against the
Lord'. He cast himself face down on the ground and remained
weeping in that position for a Jong time, confessing to God. Then
he got up and made as though to enter the sepulchre, but he could
not <enter>. The apparition of the ram prevented him no less than
before. Then the guardian of <the Chapel of> the Cross said to
him: 'There is still some other impediment', The governor replied:
'Could it be that I am forbidden to enter because I am in commu-
nion with Severus, and not with the holy catholic and apostolic
Church”? And he besought the guardian of <the Chapel of the>
Cross that he might partake of the holy and life-giving mysteries of
Christ our God. When the holy chalice arrived, he made his
communion, and thus he entered and worshipped, no longer seeing
anything <which deterred him>.

