219.
How A BROTHER WAS RECONCILED WITH
A DEACON WHO WAS AGGRIEVED AT HIM

An elder told me something like this:

Once I stayed for a short time at the Lavra of Abba Gerasimos, and
there was somebody there who was very dear to me.
One day as we
were sitting together talking about those things which are beneficial
<to the soul> I recalled this saying of Abba Poimén: that each man
should always question himself on every matter.
He said to me:
“Father, I have experience of those sayings, of their severity and of
their strength.
Once I had a beloved and dear deacon from the
lavra.
Somehow or other, something about me came to his ears
which brought him grief and he began to treat me very cooly.
When
I perceived his coldness, I sought to know the reason for it.
He said
to me: “You have done such-and-such”.
Since I was not aware of
having done any such thing whatsoever, I began to assure him, thus:
“I am not aware of having done such a thing”.
He said to me:
“Forgive me, but I am not convinced”.
I retired to my cell and
began to search my heart to see whether any such deed had been
done by me and I found nothing.
Seeing him holding the holy
chalice and distributing <holy communion> I swore to him on the

chalice that I had no knowledge of having done such a thing, but
he was not convinced.
Then I became myself again and thought of
these words of the holy fathers <that each man should always
question himself on every matter>.
I put my trust in them and
changed my line of reasoning a little.
I said to myself: “The kindly
deacon loves me and, prompted by his love for me, he has confided
to me that which was in his heart concerning me to put me on my
guard.
I will make sure that I do not do that deed in future.
But,
oh, wretched soul! While you say you have not done ¢hat deed, are
there not thousands of misdeeds done by you which you have
forgotten? Where are the things you did yesterday or the day before
that or ten days ago? Can you recall them? Is it not possible that
you have done this deed as lightly as you did the others, and have
forgotten it as readily as you forgot them?” And so I disposed my
thoughts to accept the possibility that I had in truth committed that
deed, but had forgotten it—just as I had forgotten my other
misdeeds.
Then I began to give thanks to God and to the deacon,
because, through him, God had made me worthy to acknowledge
my fault and to repent of it.
With these thoughts in my mind, I got
up to go and apologise to the deacon and to thank him, because
through him I had acknowledged my fault.
I knocked at the door;
he opened it and immediately fell at my feet, saying: “Forgive me;
I was deceived by demons into thinking that of you.
But in truth,
God has informed me that you are not guilly of anything.” He said
that he would not allow me to offer my assurance, for there was no
need.
I was greatly edified by this experience and I glorified the
Father, the Son and the Holy Spirit; to whom be the power and the
magnificence, for ever and ever.
Amen.'

