189.
THE LIFE OF A WOMAN WHO REMAINED
FAITHFUL TO HER HUSBAND, A MERCHANT,
AND HOW GOD HELPED THEM BOTH

We came too the hospice of the Fathers at Ascalon and there
Eusebios the priest said to us:

There was a merchant of our city who set sail and lost all his own
goods and everything else he was carrying at sea.
He alone was
saved, When he came back he was seized by his creditors and
thrown into prison.
Everything in his house was confiscated.
There
was nothing left to him except what he and his wife stood up in.
Although she was in great distress and anxiety, she made it a rule
at least to feed her husband with bread.
One day, as she was sitting
eating with her husband in prison, a person of note came in to
distribute some comforts to the inmates.
When he saw the woman
who was free to come and go sitting with her husband, he was
smitten with desire for her, for she was exceedingly good-looking.
He sent a message to her through the gaoler and she came to him
with a light heart, expecting to receive some charity.
He took her
aside and said to her: 'What is the matter? Why are you here? She
told him the whole story and he said to her: 'If I discharge your
debt, will you sleep with me tonight? She, who was very beautiful
and very pure-minded, said to him: 'My lord, I have heard the
Apostle <Paul; 2 Co 7:1-7> say that a wife does not have authority
over her own body: her husband has.
Let me go and ask my
husband, sir, and I shall do what he commands,' She came and told
the whole matter to her husband.
He was a wise man who loved his
wife dearly; he did not let the prospect of freedom lead him astray.
Sighing deeply and shedding tears, he said to his wife: 'Go and
refuse the man, sister, and Jet us hope in God that he will not

abandon us at the last'.
She got up and sent the man away, saying:
'I told me husband and he was unwilling'.

At that time, there was a highway man who had been thrown
into the inner prison.
Observing all that passed between the husband
and his wife, he sighed to himself and said: 'Look what a situation
they are in—yet they would not surrender their honour* <e/eu-
theria> neither for money, or to be set free.
They held chastity to
be of more worth than al] riches and they despised all the things of
this life.
And what shall I do, wretch that I am, who have never
even thought about the question of whether there is a God—and on
that account, I am responsible for so many murders.' He called
them over and through the window of <the cell> where he lay, he
said to them: 'I was a robber and many are the evil deeds and the
murders I have committed.
And for that reason, when the governor
comes and I appear before him, I shall die as a murderer.
Yet when
I saw your chastity, I was moved with compassion for you.
Go to
such-and-such a place by the city-wall; dig there, and take the
money you find.
You are to have it to discharge your debts and to
make many charitable donations.
Pray for me, that I might receive
mercy.”

A few days later, the governor came to the city and ordered the
robber to be brought out and beheaded.
The day afterwards, the
woman said to her husband: 'Is it your wish, sir, that I go to the
place revealed by the robber and see if he was telling the truth'? He
said to her: 'Do what you think best'.
She got a small mattock and
in the evening she went and dug at the spot he had mentioned.
She
found a covered pot with gold in it.
She used it very prudently,
giving it out a little at a time (as though she were borrowing from
this one and from that one) until she had discharged all <their
debts>.
Then she was also able to get her husband out.
The man
who told us this story said: 'Behold, even as they were faithful to
the law of God, so did our Lord and God multiply his mercies on
them',

