56, THE LIFE OF JOHN, THE DISCIPLE
OF A GREAT ELDER WHO LIVED IN
THE VILLAGE OF PARASEMA

Ptolemais is a city of Phoenicia. There is a village nearby called
Paraséma in which there resided a great elder. He had a disciple
named John who was also great and who excelled in obedience. One
day the elder sent the disciple to perform a task for him, giving him
a little bread to sustain him on the way. The disciple went and
completed the task and then came back, bringing the bread with
him, untouched. When the elder saw the bread, he said to him:
'Why did you not eat any of the bread I gave you, my child?”
Making an act of obeisance, the disciple said to the elder: 'Forgive
me, father, but when you blessed me and dismissed me, you did not
say I was to eat of the bread; and that is why I did not eat it'.
Amazed at the disciple's discernment, the elder gave him his
blessing. After the death of this elder, a vision from God appeared

to the brother (who had just concluded a forty-day fast) which said
to him: 'Whatever <disorder> you lay your hand on, it shall be
healed'. When morning came, by the providence of God a man
arrived, bringing his wife who had a cancer of the breast. The man
besought <the brother> to heal his wife. The brother replied: I am
a sinful man and unworthy of such an undertaking', The woman's
husband continued to beg him to accede to his request and to have
pity on his wife. So <the brother> laid his hand <on the diseased
part> and sealed <it with the sign of the cross> and she was
immediately healed. From that time on God performed many signs”
through him, not only in his own lifetime, but also after his death.*

