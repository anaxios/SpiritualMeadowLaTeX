179, THE LIFE OF A WOMAN RELIGIOUS
<SANCTIMONIALIS FEMINAE>
WHO WAS FROM THE HOLY CITy

We visited John the anchorite, known as 'the red', and he told us
that he had heard Abba John the Moabite say that there was in the
Holy City a nun <monastria> who was very devout, progressing in
the service of God. The devil resented this virgin, so he implanted
a satanic desire <for her> in the heart of a certain young man. That
wondrous virgin perceived the demon's subterfuge and <foresaw>
the young man's destruction. So she put some <beans> soaked in
water into a basket and went into the wilderness. By her withdrawal
she brought peace and serenity to the young man whilst she herself
attained the security which is borne of solitide. A long time
afterwards, by the providence of God, so that her virtuous conduct
should not remain unknown, an anchorite saw her in the wilderness
of the holy Jordan and he said to her: 'Amma, what are you doing
in this wilderness?' Not wishing to reveal herself to the anchorite,
she said to him: 'Forgive me; the fact is that I have lost my way.
But of your charity, father, and for the sake of the Lord, show me

amy path.' By divine inspiration he knew all about her. He said to
her: 'Believe me, amma, you have neither lost your way nor are you
locking for the path. You know that lies are of the devil; so tell me
the real reason why you came here'. Then the virgin said to him:
'Forgive me abba; a young man was in danger of falling into sin on
my account and, for that reason, I came into the wilderness. I
thought it was better to die here than to be an occasion of stumbl-
ing to somebody as the Apostle <Paul, 2 Co 6:3> says.' The elder
asked: 'How long have you been here?' 'Seventeen years, by the
grace of Christ', she replied. 'What do you eat?' asked the elder. She
produced the basket containing steeped <beans> and said to the
anchorite: 'I brought this basket away from the city with me
containing these few steeped <beans> and so great has been the
providence of God to me that I have been able to eat of them all
this time and they have not decreased. And this too you should
know, father: that his goodness has so sheltered me that in all these
seventeen years no man ever laid eyes on me until you did today.
Yet for my part, I could see all of them.' When the anchorite
learned this, he glorified God.

