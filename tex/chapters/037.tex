37.
THE LIFE OF A BISHOP WHO LEFT
HIS THRONE AND CAME TO THE HOLy Ciry*
WHERE HE CHANGED HIS CLOTHES AND
BECAME A BUILDER\textquotesingle S LABOURER

One of the fathers told of a bishop who left his own diocese and
came into Theoupolis, where he worked as Jabourer.
At that time
the Count of the East was Ephraim, a merciful and compassionate
man; so much so that he was rebuilding the public edifices (the city
having been dilapidated by an earthquake).
In his sleep one night
Ephraim saw the bishop lying down and a column of fire standing
over him which reached up into heaven.
As he had this vision not
once, but several times, Ephraim was greatly amazed, for it was an
awesome and truly astounding apparition, He asked himself what
it might be, for he had no idea the workman was a bishop.
How
could he have known the labourer was a bishop, in view of his
uncombed hair and shabby clothing? This was a poverty-stricken
man, broken down by much endurance, much asceticism and
labour, plus the continuous burden of much toil.
One day Ephraim
sent for the labourer who was once a bishop, to learn from him who
he was.
He took him aside and began asking him where he was
from and what his name was.
The sometime bishop said: 'I am one
of the poor men of this city.
For lack of any support I work as a
labourer and God sustains me by my toil.' God prompted Ephraim
to answer him: 'Believe me, I shall not Jet you go until you tell me
the whole truth about yourself'.
Since he could conceal himself no
longer, the bishop said to him: 'Give me your word that you will
never tell anybody what you are about to hear from me as long as
I am still alive, and I will tell you about myself.
But I will not tell
you my name or the name of my city'.
The godly Ephraim swore to
him: 'I will not tell anybody what you are about to tell me for as
Jong as it pleases God to keep you in this life'.
The other said to
him: 'I am a bishop.
At the behest of God, I left my diocese and

came to this place—because it was totally unknown to me.
Here I
have suffered affliction and laboured at menial tasks.
By my toil I
earn a little bread, but do you add what you can by way of
almsgiving.* For in these days, God is going to raise you up to the
throne of Theoupolis to be the shepherd of his people which Christ
our true God purchased by his own blood.
As I said to you, you are
to strive for almsgiving and orthodoxy.
By such sacrifices you will
be well-pleasing to God', Within a few days it came about as he had
predicted.
When the blessed Ephraim had heard the bishop out, he
glorified God saying: 'Oh, how many hidden servants God has and
they are known only to him alone'!

