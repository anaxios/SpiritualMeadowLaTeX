25.
A BROTHER AT THE MONASTERY
OF CHOZIBA, THE WORDS OF
<THE PRAYER OF> THE HOLY OFFERING
AND ABBA JOHN

Abba Gregory, a former member of the Imperial Guard, told us of
a brother at the Community of Choziba who had learned by heart
the words used at the offering of the holy gifts.* One day he was
sent to fetch the <eucharistic> oblations and, as he returned to the
monastery, he said the offering prayer—as though he were reciting
verses.
The deacons placed the same oblations on the paten in the
holy sanctuary.* The priest at that time was Abba John the Chozi-
bite who later became Bishop of Caesarea in Palestine.
When he
offered the gifts, he did not perceive the coming of the Holy Spirit*
in the accustomed manner.
He was distressed, thinking that it might
be on account of some sin on his part that the Holy Spirit was
absent.
He withdrew into the sacristy in tears and flung himself face-
down.
An angel of the Lord appeared to him and said: 'Because the
brother who was bringing the oblations here recited the holy prayer
of offering on the way, they are already consecrated and made
perfect'.
The elder laid down a rule* that from henceforth nobody
was to learn the holy prayer of offering unless he had been
ordained; nor was it ever to be recited at any time other than in a
consecrated place.

