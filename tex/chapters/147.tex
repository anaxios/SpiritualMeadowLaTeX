147, THE WONDROUS CORRECTION
OF A LETTER WRITTEN BY
'THE BLESSED ROMAN PONTIFF TO FLAVIAN*

Abba Menas, ruler of the same community,* also told us that he
had heard this from the same Abba Eulogios, Pope of Alexandria:
When I went to Constantinople, [I was a guest in the house of]
master Gregory the Archdeacon of Rome, a man of distinguished
virtue. He told me of a written tradition preserved in the Roman
church concerning the most blessed Leo, Pope of Rome. It tells
how, when he had written to Flavian, the saintly patriarch of
Constantinople, condemning those impious men, Eutyches and
Nestorios, he laid the letter on the tomb of Peter, the Prince of the
Apostles. He gave himself to prayer and fasting, lying on the
ground, invoking the chief of the disciples in these words: 'If I, a
mere man, have done anything amiss, do you, to whom the church
and the throne are entrusted by our Lord God and Saviour Jesus
Christ, set it to rights'. Forty days later, the apostle appeared to him
as he was praying and said: 'I have read it and I have corrected it'.
The pope took the letter from Saint Peter\textquotesingle s tomb, unrolled it and
found it corrected in the apostle's hand.

