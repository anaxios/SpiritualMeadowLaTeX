36, THE LIFE OF EPHRAIM, PATRIARCH OF ANTIOCH
AND HOW HE CONVERTED A STYLITE MONK FROM
THE IMPIETY OF THE SEVERAN HERESY

One of the fathers told us that the blessed Ephraim, Patriarch of
Antioch,” had a great deal of zeal and fervour for the orthodox
faith. One day he learned that a stylite in one of the regions around
Hierapolis was one of Severus' excommunicate Acephalites.* He
went to this stylite with the intention of talking him round. When
he got there the godly Ephraim began to urge and entreat the stylite
to take refuge in the apostolic throne® and to enter into communion
with the catholic and apostolic church. In answer the stylite said to
him: 'It will never be the case that I will communicate with the
<orthodox> Synod'. The godly Ephraim rejoined: 'Well then, what
have I got to do to convince you that, by the grace of Christ Jesus
our Lord, the holy Church has been set free of every trace of
heretical teaching”? The stylite said: 'Let us light a fire, my lord

Patriarch, and let you and me go into it. If one of us comes out
unharmed, he is the orthodox and he is the one we ought to follow'.
He said this to terrify the patriarch; but the godly Ephraim said to
the stylite: 'You ought to have obeyed me as a father, my child, and
to have asked nothing of us. Since you have asked something which
is beyond my meagre ability, I have put my trust in the mercies of
the Son of God that, for the sake of your soul's salvation, I will do
what you suggest'. Then the godly Ephraim said to those who stood
by; 'Blessed be the Lord! Bring some wood here'. When the wood
arrived, the patriarch lit it before the column and he said to the
stylite: (Come down and we will both walk into the fire to carry out
your test'. The stylite was amazed at the patriarch's trust in God
and he did not want to come down. The patriarch said to him: “Was
it not you who suggested we do this? How is it you no longer want
to go through with it'? Then the patriarch took off the omo-
phorion* be was wearing and, coming close to the fire, prayed in
these words: 'Lord Jesus Christ our God, who for our sakes
condescended truly to be made flesh of our Lady the holy Mother
of God and ever-virgin Mary, show us the truth', When the prayer
was finished, he threw his omopohorion into the fire. The fire
burned for three hours. Then, when the wood was all burnt up, he
retrieved the omophorion from the fire—still in one piece, It was
undamaged and unmarked and there was no sign to be found on it
of having been in the fire. When he saw what had happened, the
stylite received instruction, rejected Severus and his heresy with an
oath, and entered the holy church. He received communion at the
hands of the blessed Ephraim, glorifying God.

