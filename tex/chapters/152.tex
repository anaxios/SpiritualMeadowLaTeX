152.
THE LIFE AND SAYINGS OF MARCELLUS
THE SCETIOTE, ABBA OF THE
MONASTERY OF MONIDIA.

At the Lavra of Monidia we encountered Abba Marcellus the
Scetiote.
Wishing to edify us somewhat, he told us this:

When I was in my homeland (he was from Apameia,) there was a
charioteer there whose name was Philerémos <="'lover of the
wilderness>.
One day, when he failed to take the prize, his sup-
porters rose up, shouting: 'Philerémos takes no prize in the city'.
After I came to Scété, whenever I was tempted by my thoughts to
go to the city, I would say to myself: 'Marcellus, Philerémos takes
no victor's crown in the city', and, by the grace of Christ, that
thought kept me from leaving Scété for thirty-five years, down to
the time when the barbarians came, sold me <into slavery> and
devastated Scété.

This same Abba Marcellus told us this as though it were about
another elder who lived at Scété, but it was in reality himself: On a
certain night he got up to perform the office and, as the service was
beginning, he heard a sound like that which is made by a military
trumpet.
The elder was troubled by this and wondered to himself
from where this sound could be coming.
No soldiers were there, nor
was there any fighting in the district.
As he was pondering in this
way, behold—a demon approached him and said: 'Yes, there is war.
If you wish neither to fight nor to be attacked, go to sleep; then you
shall not be attacked.”

Again the elder said: 'Believe me, children, there is nothing
which troubles, incites, irritates, wounds, destroys, distresses and
excites the demons and the supremely evil Satan himself against us,
as the constant study of the psalms.
The entire holy Scripture is
beneficial to us and not a little offensive to the demons, but none of
it distresses them more than the psalter.
In public affairs, when one
party sings the praises of the emperor, the other party is not

distressed, nor does it move to attack the first party.
But if that
party begins reviling the emperor, then other will turn on it.
Thus
it is that the demons are not so much troubled and distressed by the
rest of holy Scripture as they are by the psalms.
For when we
meditate upon the psalms, on the one hand, we are praying on our
own account, while, on the other hand, we are bringing down curses
on the demons.
Thus, when we say Have mercy upon me O God
after your great goodness: and according to the multitude of your
tender mercies, do away with my transgressions <Ps 50:1> and
again: Cast me not away from your presence: and take not your
holy spirit from me <Ps S0:11> and Cast me not away in the time
of age: forsake me not when my strength fails me <Ps 70:9>, we are
praying for ourselves.
But then we bring down curses on the demons
when, for instance, we say: Let God arise and let his enemies be
scattered: Jet them also that hate him flee before him <Ps 61:1>, and
again: Let him scatter the people that delight in war <Ps 67:31>,
and: / myself have seen the ungodly in great power and flourishing
Jike a green bay-tree: I went by and Jo, he was gone; I sought him,
but his place could nowhere be found <Ps 36:35-36>, and Their
sword shall go through their own heart <Ps 36:15>, and He has
excavated and dug up a pit, and is fallen himself into the destruc-
tion which he made for an other <Ps 7:16-17).
His travail shall
come upon his own head and his wickedness shall fall on his own

pate'.
Again the elder said: 'Believe me, children, when I say to you

that it is a highly praiseworthy and a very glorious thing, a kingdom
in itself, for a man to take vows and become a mouk, for spiritual
pursuits are eminently preferable to the quest for what gratifies the
senses.
Therefore, great is the disgrace and the dishonour of a monk
who lays aside his habit, even if it be to become emperor.'

Again: 'In the beginning, man was in the likeness of God.
But,
when he fell away, he became like the wild beasts'.

Again: 'Nature raises up the physical desires, brethren; but the
intensification of asceticism extinguishes them'.

Again: 'You must have personal experience of the good life, and
not be frightened as though it were impossible'.

Again: 'Do not be amazed that, though you are an earthling,
you can become an angel, for a glory like that of the angels lies
before you, and he who presides over the games promises <that
glory> to those who run the race'.

Again: 'There is nothing which draws monks to God so much
as good, decent, godly purity, which is conducive to a graceful and
constant fidelity to the Lord <\ Co 7:35>.
The all-holy Spirit bears
witness to this through the godly Paul.'

Again: 'Brethren, let us leave marriage and the raising of
children to those whose eyes are towards earth, who long for the
things of the present and take no thought for that which is to come;
who do not strive to possess the good things of eternity, and are
unable to disentangle themselves from the ephemera of this world'.

Again: 'Let us make haste to depart from the life of the body,
even as Israel hurried to escape from slavery in Egypt'.

Again: 'We have the splendid and delicious rewards of God
ahead of us, brethren, in exchange for the bitter delights of this
world'.

Again, the elder said: 'Let us flee from avarice, which is the
mother of all evils' <cf1 Tm 6:10>.

