108.
THE LIFE OF A VIRGIN PRIEST AND OF
HIS WIFE, WHO WAS ALSO A VIRGIN

When we were on the island of Samos, we went to the community
named Charizenos where we met the higoumen, Abba Isidore, a
man of distinguished virtue with a great love for all humanity,
adorned with simplicity and infinite humility; later he became
bishop of the same city on Samos.* He told us this story:

About eight miles from the city <of Samos> there is an estate on
which there is a church.
It had a priest who was a very remarkable
man, His parents had forced him to marry against his will.
Not only
did this man not let himself be led into the temptation of delight

(even though he was young and legally married to the woman), he
even persuaded his wife to live with him in purity and continence.
They both learnt the psalter and they used to sing the psalms
together in church, both preserving their virginity into old age.
Now
it happened that a false accusation was made before the bishop
against this priest.
As the bishop was unaware of the true state of
affairs, he sent and brought the priest from the estate and put him
in the prison where it was customary to guard and detain clergy
who had gone astray.
Whilst he was in the prison, as the holy day
of the Lord was dawning but whilst it was still night, there appeared
to him an extremely handsome young man who-said to him: 'Priest,
arise: be off to your church and celebrate the eucharist'.
The priest
said to him: 'I cannot, for I am a prisoner'.
The apparition said to
him: 'T will open the prison.
Come, follow me.' He opened the door
of the prison and led the way out.
When he was out, he accom-
panied the priest to within a mile of the estate.
After the break of
day, the jailor went in search of the prisoner and, when he could
not find him, he went to the bishop, saying: 'He has run away from
me and I had the key!” Thinking that he had indeed run away, the
bishop sent one of the episcopal servants, saying: \textquotesingle Go and see if that
priest is on his estate—but do not take any further action against
him', The servant went and found the priest in the church, celebrat-
ing the eucharist.
He returned and said to the bishop: 'He is there,
and I saw him celebrating the eucharist'.
The bishop became even
more angry with the priest and swore to bring him back in dishon-
ourable custody next day.

The night preceding Monday, the vision he had seen earlier
appeared again to the priest, saying to him: 'Come along, we must
return to that place in the city into which the bishop cast you'.
He
took the priest and led him back again, replacing him in the prison
without the knowledge of the man who was charged with responsi-
bility for it.
At daybreak on Monday the bishop learned from this
man that (without his knowing how it had come about) he had

found the priest back in gaol.
The bishop sent for the priest and
demanded of him how he had got out of the prison and then come
in again without the knowledge of the gaoler.
This was the priest\textquotesingle s
reply: 'A very handsome young servant, beautifully dressed, who
said he was of the episcopal retinue—he opened up for me and led
me to within a mile of the estate on Saturday night.
He came to me
again last night and brought me back'.
The bishop brought forward
all the episcopal servants but the priest did not recognise one of
them.
Then the bishop realised that il was an angel of God who had
done this deed, so that the virtue of the priest should not be entirely
concealed—but that all might learn of it and glorify the God who
glorifies his servants.
He dismissed the priest in peace whilst
complaining bitterly against those who had falsely accused him.

