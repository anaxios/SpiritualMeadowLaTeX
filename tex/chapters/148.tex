148.
THE VISION OF THEODORE,
BISHOP OF DARA,
CONCERNING THE SAME MOST BLESSED LEO

Theodore, the most holy bishop of the city of Dara in Libya, told
us this:

When I was syncellos to the saintly Pope Eulogios, in my sleep I
saw a tall, impressive looking man who said to me: 'Announce me
to Pope Eulogios'.
I asked him: 'Who are you, my lord? How do
you wish to be announced” He replied: 'I am Leo, Pope of Rome',
so I went in and announced: 'The most holy and most blessed Leo,
Primate of the Church of the Romans, wishes to pay you his
respects'.
As soon as Pope Eulogios heard, he got up and came
running to meet him.
They embraced each other, offered a prayer
and sat down.
Then the truly godly and divinely-inspired Leo said
to Pope Eulogios: 'Do you know why I have come to you”? The
other said he did not.
'I have come to thank you', he said, 'because
you have defended so well, and so intelligently, the letter which I
wrote to our brother, Flavian, Patriarch of Constantinople.
You
have declared my meaning and sealed up the mouths of the heretics,
And know, brother, that it is not only me whom you have gratified
by this labour of yours, but also Peter, the chief of the apostles;
and, above all, the very Truth which is proclaimed by us, which is
Christ our God.' I saw this, not only once, but three times.
Con-
vinced by the third apparition.
I told it to the saintly Pope Eulogios.
He wept when he heard it and, stretching out his hands to heaven,
he gave thanks to God, saying: 'I give you thanks, Lord Christ, our
God, that you have made my unworthiness become a Proclaimer of
the truth, and that, by the prayers of your servants Peter and Leo,
your Goodness has received our feeble endeavour <as you did
Teceive> the widow's two mites.'

