94, THE LIFE OF ABBA JULIAN, THE
BISHOP OF BOSTRA

This same father of ours, George the archimandrite, also told us
about Abba Julian who became bishop of Bostra: that after he left
the community and became bishop of Bostra, certain affluent
citizens who were enemies of Christ wanted to do away with him by
poison.
They corrupted his butler with money and gave him some
poison to drop into the cup when he poured out a drink for the
metropolitan <bishop>.
The servant did as they told him.
When he
had given the poisoned cup to the godly Julian, the bishop received
it but, by divine inspiration, he knew of the conspiracy and of those
who had perpetrated it.
So he took the cup and set it down in front
of himself without saying a word to the servant.
He sent and
summoned all the chief citizens, amongst whom were those who had
engineered this conspiracy against him.
Now the godly Julian did
not wish to make a public disgrace of the guilty ones.
He said to
them all in a gentle voice: 'If you thought you could destroy the
humble Julian with poisons, look; I will drink this in full view of
you all', He made the sign <of the cross> three times over the cup
with his finger and with the words, 'I drink this cup in the name of
the Father and of the Son and of the Holy Ghost', he drank it
down before them all—and remained unharmed.
When they saw
this, they cast themselves down before him in an act of repentance.

