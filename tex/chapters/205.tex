205. CONCERNING ANOTHER WISE WOMAN
WHO, BY JUDICIOUS ADVICE,
TURNED ASIDE A MONK WHO WAS HARASSING HER

Somebody told of a brother who lived in a community and who
used to be sent to conduct the business of his house. There was a
devout secular person in a village who used to give him hospitality
as an act of faith, as often as he came in and out of the village.
<This man> had a daughter who had recently been widowed after
living with her husband for a year or two. As the brother came in
and out of their house, he began to be troubled by thoughts of her.

As she was no fool, she realised this, and took care not to enter his
presence. One day her father went into the neighbouring city on
necessary business, leaving her alone in the house. The brother
came, as was his custom, and finding her alone in the house, he said
to her: 'Where is your father?' 'He has gone into the city', she
replied. Then he began to be troubled by temptation and wanted to
throw himself on her. She prudently said to him: 'Do not be
troubled; my father shall not return until evening; there are <only>
the two of us here. But I know you monks never do anything
without prayer. Get at it, then; pray to God, and if he puts it in
your heart to do something, that we will do'. This was not accept-
able to him, for temptation continued to rage within. She said to
him: 'Have you ever really known a woman?' He said: 'No; and that
is why I want to know what it is like'. She said to him: 'That is why
you are troubled by temptation, for you do not know the bad odour
of wretched women'. And to cool his ardour, she added: 'I am
having my period. Nobody can come near me or bear the smell of
me for the stench which mars my body'. When he heard this and
similar things from her, he regretted what had happened; he became
himself again, and wept. When she realised that he was his normal
self again, she said: 'Look, if I had listened to you and given in to
you, we would already have been satisfied—and would have sinned
utterly. How then could you have looked my father in the face or
gone back to your monastery and heard the choir of those holy ones
who sing <there>? Be sober, I beg of you; and do not be so ready
to lose all the sufferings you have endured and to deprive yourself
of the good things of eternity, just for the sake of a little short-lived
pleasure.' When he heard what she had to say, the brother to whom
it happened told it to him who now related <the story>, giving
thanks to God who, by the woman's prudence and temperance, had
prevented him from taking an irremediable fall.

