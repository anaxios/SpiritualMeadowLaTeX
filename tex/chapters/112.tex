112.
THE LIFE AND DEATH OF LEO,
A CAPPADOCIAN MONK

In the reign of the Emperor and most faithful Caesar, Tiberius,* we
went to <the Great> Oasis and when we were there, we saw a
monk, a Cappadocian by race, who was great in the eyes of God.
Many people told us a multitude of wondrous stories about this
monk, When we made contact with him and gained some experience
of him, we reaped considerable benefits; especially from the

humility, the recollection, the poverty and the charity which he
showed to all.* This ever-memorable elder said to us: 'Believe me,
children, I am going to reign'.
We said to him: 'Believe us, abba,
nobody from Cappadocia ever reigned;* this is an ill-suited thought
you are harbouring'.
But he said again: 'It is a fact, children, that
I am going to reign', and nobody could persuade him to put the
idea away from him.

When the Maziques* came and overran all that region, they
came to <the Great> Oasis and slew many monks, while many
others were taken prisoner.
Among those taken prisoner at the
Lavra <of the Great Oasis?> were Abba John, formerly lector at the
Great Church in Constantinople, Abba Eustathios the Roman, and
Abba Theodore, al] three of whom were sick.
When they had been
captured, Abba John said to the barbarians: 'Take me to the city
and I will have the bishop give you twenty-four pieces of gold'.
So
one of the barbarians led him off and brought him near to the city.
Abba John went in to the bishop.
Abba Leo was in the city at that
time and so were some others of the fathers; that is why they were
not captured.
Abba John went in and began to implore the bishop
to give the barbarian the twenty-four pieces of gold, but the bishop
could only find eight.
He was willing to give these to the barbarian,
but he would not take them.
'Either give me twenty-four pieces of
gold or the monk', he said.
The men of the fortress had no choice
but to hand over Abba John (who wept and groaned) to the barbar-
ian; they took him away to their tents.
Three days later, Abba Leo
took the eight pieces of gold and went out into the wilderness to
where the barbarians were camped.
He pleaded with them in these
words: 'Take me and these eight pieces of gold, and let those <three
monks> go.
For, as they are sick and cannot work for you, you will
only have to kill them.
But as for me, I am in good health and I can
work for you'.
Then the barbarians took both him and the eight
pieces of gold of which he spoke, letting the other three <monks>
go free.
Abba Leo went off somewhere with them and when he was

exhausted [and could go no further], they beheaded him.
And Abba

Leo fulfilled that which is spoken in the scriptures: Greater Jove
hath no man than this, that a man lay dowa his life for his friends

<Jn 15:13>.
Then we knew what he was talking about when he used
to say: 'I am going to reign', for reign he did, having Jaid down his
life for his friends.

