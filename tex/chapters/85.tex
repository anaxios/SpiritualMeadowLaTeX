85. HOW THE WHEAT OF THE SAME MONASTERY
GERMINATED BECAUSE THE CUSTOMARY
ALMSGIVING HAD BEEN SUSPENDED

They also told us this:

It used to be the custom for the poor and the orphans of the region
to come here on Maundy Thursday to receive half a peck of grain
or five loaves of blessed bread, five small coins, a pint of wine and
half a pint of honey. For three years prior to this happening <which
we are about to tell>, grain had been scarce and in this area it was
selling at one piece of gold for two pecks. When Lent came round,
some of the brethren said to the higoumen: 'Abba, do not make
provision for the customary dole to the poor this year, sir, lest the
monastery not have enough for the brethren—for grain is not to be
found', The abba began to say to the brethren: 'Children, let us not
discontinue the charity of our father <Theodosios>. Behold, it is his
commandment and it would be held against us if we disobeyed it.
It is he himself who will look after us'. But the brethren continued
to argue with the abba, saying: 'We cannot give the accustomed
charity for we do not have anything to give'. Then the higoumen
was deeply grieved <but he said to them> 'Go then and do what
you will'. The customary charity therefore was not distributed
<that> Maundy Thursday. But on Good Friday morning, the

brother in charge of the granary opened up and found that what
grain they did possess had germinated. So they ended up throwing
it all into the sea. Then the abba began to say to the brethren: 'He
who sets aside the commandments of his father suffers these
afflictions. You are now reaping the fruits of disobedience. We were
going to part with five hundred pecks <=125 bushels> <of grain>
and, in doing so, to serve our father Theodosios by <our> obedi-
ence. Also to bring consolation to our brethren <the poor.> Now
about five thousand pecks <=1250 bushels> of grain has gone to
ruin, what good has it done us, brethren? We have twice been guilty
of wrong-doing: once in that we transgressed the precept of our
father, and again in that we did put our trust not in God, but in our
granary. So let us learn from this <experience,> my brethren, that
God watches over all humanity; and that Saint Theodosios invisibly
cares for us, his children'.

