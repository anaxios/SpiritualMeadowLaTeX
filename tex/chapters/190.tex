190. THE MIRACLE OF SOME WOOD GIVEN TO
ABBA BROCHA, THE EGYPTIAN

Athanasios the Egyptian, who was connected with the civil author-
ity,* said that Abba Brocha found a spot in the wilderness outside
the city of Seleucia near Antioch and tried to build a small cell
there. As his building progressed, he wanted wood to build the roof.
One day he went into the city and found Anatolios, known as 'the
hunchback', a magnate of Seleucia, sitting outside his house. He
went up to him and said: 'Of your charity, give me a little wood to
roof my house with. The magnate replied testily: 'Look, there is
wood over there; take it, and go', and he indicated a large mast
which he had lying in front of his house and which he had made for
a vessel of fifty-thousand bushels. Abba Brochas said: '<The Lord>
bless you; I will take it', Still in a bad humour, Anatolios said:
“Blessed be God'. <The elder> grasped the mast, lifted it from the
ground all by himself and put it on his shoulders, In this way he
took it away to his cell. Anatolios was so taken aback by this
extraordinary miracle that he granted him as much wood as he
required for his needs. With this, Abba Brocha was able not only to
roof his cell of which we spoke, but to do many other things for his
monastery.

