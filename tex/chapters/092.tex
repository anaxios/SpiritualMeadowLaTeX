92, THE LIFE OF BROTHER GEORGE
THE CAPPADOCIAN AND THE FINDING OF THE BODY
OF PETER THE SOLITARY OF THE HOLY JORDAN

Our holy father, Abba George, archimandrite of the monastery of
our holy father Theodosios which lies in the wilderness of the Holy
City of Christ our God, told this to me and to brother Sophronios
the Sophist:

I had a brother here known as George the Cappadocian. He used
to do manual work at Phasaelis. One day when the brothers were
making loaves of bread, brother George was heating the oven. But
when he had heated the oven he could not find the implement for
wiping it out—because the brethren had hidden it to put him to the
test. So he went in <to the oven>* and wiped it out with his
garment. And he came out again not in the least harmed by the fire.
When I heard of this I reproved the brethren for putting him to the

test.
The same abba, our father George, also told us this about the

same brother George:
One day he was pasturing swine in Phasaelis when two lions came

to seize a pig. He took up his staff and chased them as far as the
holy Jordan.

Again this same father of ours spoke to us saying:
When I was about to build the Church of Saint Kerykos at

Phasaelis, they dug the foundations of the church and a monk, very
much an ascetic, appeared to me in my sleep. He wore a tunic of

sack-cloth and on his shoulders he had an over-garment made of
rushes. In a gentle voice he said to me: 'Tell me, Abba George, did
it really seem just to you, sir, that after so many labours and so
much endurance, I should be left outside the church you are
building? Out of respect for the worth of the elder, I said to him:
'Who in fact are you, sir”? He said: \textquotesingle I am Peter the grazer of the
holy Jordan.' I arose at dawn and enlarged the plan of the church.
As I dug, I found his corpse lying there, just as I had seen him in
my sleep. When the oratory was built, I constructed a handsome
monument in the right-hand aisle, and there I interred him.

