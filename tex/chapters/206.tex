206. A STRATAGEM BY WHICH A GREAT
LADY WAS TAUGHT HUMILITY

One of the holy fathers said that a woman of senatorial rank came
to worship at the Holy Places. When she came to Caesarea, it
pleased her instead to stay there in solitary retirement <Aésuchasai>.
She asked the bishop to give her a virgin who could train her in
religion and teach her the fear of God. The bishop selected a
modest <virgin> and gave her to <the great lady>. Sometime later
the bishop encountered her and asked: 'How is the virgin I gave
you?' 'She is fine', she replied, 'but not much benefit to my soul
because she is so humble that she lets me go my own way. I need
somebody who will stand up to me and not let me do whatever I
want'. So the bishop took away the first virgin and sent another, a
stern one who used to her address her as 'fool of a rich woman' and
heap similar imprecations upon her. Afterwards, the bishop asked
her again how she found the virgin and the lady replied: 'This one
is certainly good for my soul', and she became distinguished for her
humility.

