144, INJUNCTIONS OF ONE OF THE ELDERS
WHO WERE AT THE CELLS

One of the elders said to the brethren at The Cells: 'Let us not
enslave ourselves to the pleasures of Egypt which <deliver us into
the hands of> the wicked tyrant, Pharaoh'.

Again: 'If only people would care as much for good things as
they care about that which is bad. If only they would transfer to a
yearning for piety all the attention they lavish on spectacles,
magnificent festivals, on avarice, vain-glory and injustice. We are
not ignorant of how highly God values us, nor are we powerless
against the demons.”

Again: 'Nothing is greater than God; nothing is equal to him;
nothing is only a little inferior to him. What then is stronger or
more blessed than someone who has the help of God?”

Again: 'God is everywhere. He draws near to those who live
devoutly and fight the spiritual battle; to those whose religion goes
further than mere pronouncements, but who are distinguished by
their deeds. Where God is present, who would wish to hatch
conspiracies? Who would be strong enough to inflict any hurt?'

Again: 'The strength of man does not lie in his physical
constitution, for that is subject to change. It lies rather in his
intention, assisted by God. Let us therefore care for our souls as we
do for our bodies, children.'

Again: 'Let us gather together the cures of the soul: piety,
righteousness, humility, submission. The greatest physician of souls,
Christ our God, is near to us and is willing to heal us; let us not
under-estimate him.'

Again: 'The Lord teaches us to be sober but, wretches that we
are, by soft-living we become yet more addicted to the delights <of
the flesh>.”

Again: 'Let us offer ourselves to God, as Saint Paul says, as
Jiving men returned from the dead <Rm 6:13>, neither looking back

or remembering what has gone before, but pressing toward the
imark for the prize of the high calling <Ph 3:14>.

A brother asked the elder: 'Why am I always sitting in judgment
on my brothers?' The elder replied: 'Because you do not yet know
yourself. Someone who knows himself does not see the <short-

comings> of his brothers.'

