105.
THE LIFE OF A HOLY ELDER NAMED
CHRISTOPHER, A ROMAN

'When we were in Alexandria, we visited Abba Theodoulos who was
at <the Church of> Saint Sophia <= 'holy wisdom\textquotesingle > by the
Lighthouse.
He told us:

It was in the community of our saintly father Theodosios (which is
in the wilderness of the city of Christ our God) that I renounced the
world.
There I met a great elder named Christopher, a Roman by
race.
One day I prostrated myself before him and said: 'Of your
charity, abba, tell me how you have spent your life from youth up'.
As I persisted in my request and because he knew I was making it
for the benefit of my soul, he told me, saying: When I renounced
the world, child, I was full of ardour for the monastic way of life.
By day I would carefully observe the rule of prayer; and at night I
would go to pray in the cave where the saintly Theodosios and the
other holy fathers were buried.
As I went down into the cave, I
would make a hundred prostrations to Gad at each step: there were
eighteen steps.
Having gone down all the steps, I would stay there
until they struck the wood<-en signal> <for matins,> at which time
I would come back up for the regular office.
After ten years spent
in that way, with fastings and continence and physical labour, one
night I came as usual to go down into the cave.
After I had
performed my prostrations on each step, as I was about to set foot
on the floor of the cave, I fell into a trance.
I saw the entire floor
of the cave covered with lamps, some of which were lit and some
were not.
I also saw two men, wearing mantles and clothed in white,
who tended those lamps.
I asked them why they had set those lamps
out in such a way that we could not go down and pray.
They

replied: 'These are the lamps of the fathers'.
I spoke to them again:
“Why are some of them lit while others are not'? Again they
answered me: 'Those who wished to do so lit their own lamps'.
Then I said to them: 'Of your charity, <tell me:> is my lamp lit or
not? 'Pray', they said, 'and we will light it'.
'Pray? I immediately
retorted, 'and what have I been doing until now?' With these words
I returned to my senses and, when I turned round, there was not a
person to be seen.
Then I said to myself: 'Christopher, if you want
to be saved, then yet greater effort is required'.
At dawn I left the
monastery and went to Mount Sinai.
I had nothing with me but the
clothes I stood up in.
After I had spent fifty years of monastic
endeavour there, a voice came to me saying: 'Christopher,
Christopher, go back to your community in which you fought the
good fight, so that you may die with your fathers'.
And a little
while after he told me this, bis holy soul went joyfully to rest in the
Lord.

Again, the same Abba Theodoulos told us about this Abba
Christopher.
According to him, the elder said: 'One day I went up
from the monastery to the Holy City to venerate the Holy Cross.
After I had performed my devotions, as I was coming out of the
ante-chamber of the Holy Cross, I saw a brother <standing> at the
door, neither going in nor coming out.
I also saw two ugly crows
flying in his face and brushing their wings against his eyes, effective-
ly preventing him from entering the shrine.
Knowing them to be
demons, I said to him: “Tell me, brother, why do you hesitate in the
doorway itself and not go in”? He said: “Forgive me, abba; I have
conflicting emotions, sir.
One urges me to enter and to venerate the
honourable Cross, but the other says: “No; make an excuse and
make your devotions some other time'”, When I heard this, I took
him by the hand and led him into the shrine; the crows immediately
fled from him.
I got him to venerate the Holy Cross and the Holy
Sepulchre* of Christ our God, then I dismissed him in peace.
The
elder told me these things (he explained) because he could see that

I was much distracted by my duties and he perceived that I was
neglecting my prayers.

