218.
How ABBA SERGIOS PACIFIED A CURSING
FARMER BY PATIENCE

The higoumen of the monastery of Abba Constantine, Abba
Sergios, told us:

Once we were travelling with a holy elder and we lost our way.
Quite without meaning to, and indeed without knowing where we
were going, we found ourselves in sowed fields and we trod down
some of the seedlings.
The farmer was working there and he noticed
what we had done.
He began to upbraid us angrily in these words:
'You are monks? You fear God? If you had the fear of God before
your eyes you would not have done this.' At once the holy elder

said to us: 'For the Lord's sake, let nobody say anything', and he
addressed the farmer: 'Well spoken, my child.
If we had the fear of
God, we would rot be doing these things.' Again the farmer spoke
angry and abusive words, to which the elder again responded: “You
speak the truth, child, when you say that if we were true monks we
would not have done this; but, for the sake of the Lord, forgive
<us>, for we have sinned', The farmer was astonished.
He came and
threw himself at the feet of the elder, saying: 'I have sinned, forgive
me, and for the Lord's sake, take me with you'.
The blessed Sergios
said: 'And in truth he followed along with us; and when he came
<here> he received the monastic habit'.

