203, THE STORY OF A JEWELLER WHO,
BY A WISE DECISION, SAVED HIS LIFE AT SEA

One of the fathers said there was a jeweller of the kind known as a
gem-engraver. He had some very valuable stones and pearls when
he went aboard a ship together with his servants, it was his
intention to go do business elsewhere. By the providence of God, it
happened that he became very fond of the member of the ship's
crew who was detailed to wait upon him. This servant slept near
him and ate the same food as he ate. One day this boy heard the
sailors whispering to each other and deciding among themselves to
throw the gem-engraver into the sea, to get their hands on the
stones he had with him. It was a very disturbed servant who went
in to wait on the good man* as usual. 'Why are you so subdued
today, boy? asked the jeweller, but the other kept his counsel and
said nothing. He asked him again: 'Come now, tell me what is
matter', at which the servant broke down into tears and sobbed out
that the sailors were planning to do this and that to the jeweller,
who asked: 'Is this really so?' 'Yes' was the reply; 'That is what they
have decided among themselves to do to you.' Then the jeweller
called his servants and said to them: 'Whatever I tell you to do, do
it at once and without arguing'. Then he unfolded a linen cloth and
said to them: 'Bring the inlaid chests', and they brought them. He
opened them and began taking out the stones. When they were all
set out, he began to say: 'Is this what life is <all about>? Is it for
these that I put my life in danger and at the mercy of the sea when,
in a little while, I shall die, and take nothing with me out of this
world? He said to his servants: 'Empty it all into the sea'. As soon
as he spoke, they cast the riches into the sea. The sailors were
amazed—and their conspiracy was frustrated.

