201.
THE LIFE OF A MOST NOBLE MAN
OF CONSTANTINOPLE WHOSE FATHER,
WHEN HE WAS DYING, LEFT HIM
THE LORD JESUS CHRIST AS HIS GUARDIAN

One of the fathers who had gone to Constantinople to attend to
some necessary business said:

Whilst I was sitting in the church, a man who was illustrious in the
worldly sense but <also> a great lover of Christ came in; and when
he saw me, he sat down.
He then began asking about the salvation
of the soul.
I told him that the heavenly life is given to those who
live the earthly life in a seemly way.
'You have spoken well, father',
he said.
'Blessed is the man whose hope is in God and who presents
himself as an offering to God.
I am the son of a man who is very
distinguished by the standards of the world.
My father was very
compassionate and distributed huge sums amongst the poor.
One
day he called me; showing me all his money, he said to me: “Son,
which do you prefer; that I leave you my money, or that I gave you
Christ as you guardian?” Grasping the point he was making, I said
I would rather have Christ; for everything that is here today shall
be gone tomorrow: Christ remains for ever.
So from the moment he
heard me say that, he gave without sparing, leaving very little for
me when he died.
So I was left a poor man and I lived simply,
putting my hope in the God whom he bequeathed to me.
There was
another rich man, one of the leading citizens, who had a wife who
loved Christ and feared God; and he had one daughter: his only
child.
The wife said to the husband: “We have only this one daugh-
ter, yet the Lord has endowed us with so many goods.
What does

she lack? If we seek to give her <in marriage> to somebody of our
own rank whose way of life is not praiseworthy, it shall be a
continual source of affliction to her.
Let us rather look for a lowly
man who fears God; one who will love her and cherish her accord-
ing to God's holy law”.
He said to her: “This is good advice.
Go to
church and pray fervently.
Sit there, and whoever comes in first, he
it is whom the Lord has sent.” This she did.
When she had prayed,
she sat down and it was I who came in at that moment.
She sent a
servant to call me straightaway and she began asking me where I
was from.
I told her that I was from this city, the son of such-and-
such a man.
She said: “He who was so generous to the poor? And
have you a wife?” I said I had not.
I told her what my father had
said to me and what I had said to him.
She glorified the Lord and
said: “Behold, the good guardian whom you chose has sent you a
bride—and riches, so that you may enjoy both in the fear of God”.
I pray that I might follow in my father's footsteps to the end of my

days.

