195. THE LIFE OF EVAGRIOS THE PHILOSOPHER
WHO WAS CONVERTED TO THE CHRISTIAN FAITH BY
SYNESIOS, BISHOP OF CYRENE

While we were in Alexandria, Leontios of Apamea, a devout man
who loved Christ, came from Pentapolis (where he had made his
home for some years at Cyrene). In <those> days of <Eulogios>,
the saintly Pope of Alexandria, the future bishop> of the same town
of Cyrene came <too>. And when we were all together, he told us
this:
That in the time of Theophilos, the blessed Pope of Alexandria,
Synesios the philosopher became Bishop of Cyrene. When he came
to Cyrene, he found there a philosopher named Evagrios who had
been his fellow student and had remained his good friend, even
though he was strongly attached to the cult of idols. Bishop
Synesios wanted to convert him. He not only wanted to, but also
made great efforts and put himself to much trouble and care for the
sake <of the friendship> in which he held him from the beginning.
The other would neither be persuaded, nor would he in any way
accept the bishop's teaching. Yet, for the sake of his great friendship
for him, the bishop was unfagging in his efforts, continuing day by
day to instruct, entreat and exhort his friend to believe in Christ and
to come to full knowledge of him. And it had this effect: that one
day the philosopher said to him: 'You know, Bishop, of all the
things which you Christians say, there is this, sir, which displeases
me. It is that there will be an end to this world and that, after the
end, everybody who existed throughout this age shall arise in this
human body and shall live for ever in that incorruptible and
immortal flesh; that they shall receive their rewards, a body who has
compassion on the poor lends to God; that anyone who distributes
money to the poor and destitute lays up treasures in heaven and
shall receive them back from Christ an hundredfold at the regener-
ation, together with eternal life. All this seems to me to be deception

and a laughing matter; a yarn which is no more than an old wives'
tale.' Bishop Synesios assured him that all the <beliefs> of the
Christians were true and that there was nothing false or alien to the
truth about them. He attempted to demonstrate with many examples
that this was so.

A long time afterwards, the bishop succeeded in making him a
Christian. He baptised the philosopher, his children and everybody
in his household. A little while after his baptism, he gave the bishop
three gold denarii for the benefit of the poor. 'Take these three
kenténaria, give then to the poor and let me have a certificate that
Christ shall give them back to me in the world to come'. The bishop
took the gold and promptly made out the desired certificate. The
philosopher lived for some years after his baptism, and then he fell
terminally ill. At the point of death, he said to his children: 'When
you prepare me for burial, put this paper in my hands and bury me
with it', When he died, they did as he had commanded and buried
him together with the hand-written paper. The third day after his
burial, while Bishop Synesios was lying down at night, the philos-
opher appeared to him and said: 'Come to the tomb where I lie and
take your hand-written paper, for I have received what was owing
to me. I am satisfied and I have no further claim on you. To make
you quite sure, I have counter-signed the paper in my own hand.'
The bishop was not aware that his hand-written certificate had been
buried with the philosopher.

The next morning he sent for <the dead man's> sons and said
to them: 'What did you deposit in the tomb together with the
philosopher? They thought he was speaking to them about money
and they replied: 'Nothing, my lord, except the grave clothes'. 'What
then', he asked; 'Did you not bury a paper with him?' Then they
remembered, for they did not realise he was talking about a paper.
They said: 'Yes, my lord; when he was dying, he gave us a paper
and said: “When you prepare me for burial, lay me out that I am
holding this paper in my hand, and nobody else is to know about

it.”' Then the bishop told them of the dream he had seen that night.
He took the sons, the clergy and some prominent citizens and went
off to the philosopher's tomb. They opened it, and they found the
philosopher lying there, holding the bishop's hand-written certificate
in his own hands. They took it from his hands, opened it and found
this, newly written on it, in the philosopher\textquotesingle s hand: 'From me,
Evagrios the Philosopher, to you sir, the most holy Bishop Synesios,
greetings. I have received what you wrote down in this promissory
note. I am satisfied and I have no further claim on you in respect
to the gold which I gave you; or rather, by your agency, to Christ
our God and Saviour.' Great was the amazement of those who saw
it. For many hours they cried out: 'Lord have mercy', glorifying
God who works wonders and grants such assurance to his servants.
Master Leontios assured us that the manuscript with the philoso-
pher's signature has survived to this day and that it is lying in the
treasury of the church of Cyrene. It is delivered into the safe-
keeping of each man who is appointed custodian there, together
with the sacred vessels. He guards it diligently and will pass it on,
safe and sound, to his successors.

